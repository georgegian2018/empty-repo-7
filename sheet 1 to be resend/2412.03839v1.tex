%%%%%%%%%%%%%%%%%%%%%%%%%%%%%%%%%%%%%%%%%%%%%%%%%%%%%%%%%%%%%%%%%%%%%%%%%%%%%%%%%%%%
%%----------------------------------------------------------------------------------
% DO NOT Change this is the required setting A4 page, 11pt, onside print, book style
%%----------------------------------------------------------------------------------
\documentclass[a4paper,11pt,oneside]{article} 
%\usepackage{CS_report} % DO NOT REMOVE THIS LINE. 
%%%%%%%%%%%%%%%%%%%%%%%%%%%%%%%%%%%%%%%%%%%%%%%%%%%%%%%%%%%%%%%%%%%%%%%%%%%%%%%%%%%%
\usepackage[english]{babel}
% \usepackage{algorithm}
% \usepackage{algorithmic}
\usepackage{dirtytalk}
\usepackage[utf8]{inputenc}
\usepackage{graphicx}
\usepackage{amssymb,amsmath,amsthm,amsfonts}
\usepackage{multicol}
\usepackage{multirow} % used for tables to merge multiple rows
% \usepackage[table]{xcolor}
\usepackage{float}

\usepackage{lettrine,paralist,fancyhdr}
\usepackage{nomencl,makeidx,pdfpages}
\usepackage{setspace}

\usepackage{longtable}
\usepackage{booktabs} % used for tables

\usepackage{bigdelim} % used for tables to set spacing
\usepackage{bigstrut} % used for tables to set spacing
\usepackage{tabularray}
\usepackage{tabularx}
\usepackage{ragged2e}

% Set page size and margins
% Replace `letterpaper' with `a4paper' for UK/EU standard size
% \usepackage[letterpaper,top=2cm,bottom=2cm,left=3cm,right=3cm,marginparwidth=1.75cm]{geometry}

\usepackage{float}
 \let\Bbbk\relax

%\DeclareUnicodeCharacter{2212}{-} %It's for the negative sign in exponent to be shown
\usepackage[table,xcdraw]{xcolor}  


\usepackage[labelfont=bf]{caption}
\usepackage{subcaption}
\captionsetup[subfigure]{labelformat=simple, labelsep=colon}
\renewcommand{\thesubfigure}{(\alph{subfigure})}
\usepackage{epstopdf}
%\usepackage{comment}
\usepackage{textcase}

%%%%%%%%%%%%%%%%%%%%%%%%%%%%%%%%%%%%%%%%%%%%%%%%%%%%%%%%%%%%
\usepackage{listings}
\usepackage{xcolor}
\lstset{
  basicstyle=\footnotesize\ttfamily,  % Adjust font size for fitting code
  keywordstyle=\color{blue},
  commentstyle=\color{gray},
  stringstyle=\color{red},
  numbers=left,
  numberstyle=\tiny,
  stepnumber=1,
  breaklines=true,     % Automatically break long lines
  tabsize=4,
  captionpos=b,
  frame=single
}

\lstset{
  language=VHDL,
  basicstyle=\ttfamily\footnotesize,
  keywordstyle=\bfseries\color{blue},
  commentstyle=\color{purple},
  breaklines=true
}

\lstset{
    basicstyle=\ttfamily, % Monospaced font for the code
    frame=single, % Single line border around the listing
    xleftmargin=0.5cm, % Adjust the left margin (resize box from the left)
    xrightmargin=0.5cm, % Adjust the right margin (resize box from the right)
    aboveskip=0.5cm, % Add space above the listing
    belowskip=0.5cm, % Add space below the listing
    breaklines=true % Enable line breaking for long lines
}

\lstset{language=Matlab, 
    basicstyle=\ttfamily\footnotesize, 
    keywordstyle=\color{blue}\bfseries, 
    commentstyle=\color{purple},  % Comments in purple
    stringstyle=\color{brown}, 
    numbers=left, 
    numberstyle=\tiny\color{gray}, 
    stepnumber=1, 
    numbersep=10pt, 
    frame=single, 
    breaklines=true, 
    breakatwhitespace=false, 
    showspaces=false, 
    showtabs=false, 
    tabsize=2,
    captionpos=b
}






% Define Maple language with MATLAB-style aesthetics
\lstdefinelanguage{Maple}{
    morecomment=[l]{\#},           % Line comments start with #
    morestring=[b]'',               % Strings are enclosed in double quotes
    sensitive=false                % Case-insensitive
}

% lstset configuration for MAPLE
\lstset{
    language=Maple,                       % Use the custom Maple language
    basicstyle=\ttfamily\footnotesize,    % Monospace font, small size
    commentstyle=\color{red},             % Comments in red
    stringstyle=\color{brown},            % Strings in brown
    numbers=left,                         % Line numbers on the left
    numberstyle=\tiny\color{gray},        % Line number style
    stepnumber=1,                         % Increment of line numbers
    numbersep=10pt,                       % Distance from line numbers to code
    frame=single,                         % Frame around the code block
    breaklines=true,                      % Break long lines automatically
    breakatwhitespace=false,              % Do not restrict breaks to spaces
    postbreak=\mbox{\textcolor{red}{$\hookrightarrow$}\space}, % Continuation marker
    showspaces=false,                     % Do not show spaces explicitly
    showtabs=false,                       % Do not show tabs explicitly
    tabsize=4,                            % Tab size equivalent to 4 spaces
    captionpos=b,                         % Caption position at the bottom
    xleftmargin=10pt,                     % Left margin for the code block
    framexleftmargin=5pt                  % Inner left margin of the frame
}



%%%%%%%%%%%%%%%%%%%%%%%%%%%%%%%%%%%%%%%%%%%%%%%%%%%%%%%%%%%%

\usepackage{pgfplots}
\usepackage{pgfplotstable}
\usepackage{tikz}
% Enable compatibility mode (required to avoid errors)
\pgfplotsset{compat=newest}
\pgfplotsset{compat=1.18}
\usetikzlibrary{3d}
\usepackage{eqparbox}

\usepackage{tikz}
\usepackage{tikz-3dplot}
\usetikzlibrary{positioning}
\usetikzlibrary{shapes.geometric, calc}
\usetikzlibrary{shapes.geometric, arrows}
\usetikzlibrary{shapes.geometric, arrows.meta}

%\usetikzlibrary{positioning}
%\usetikzlibrary{shapes.geometric, arrows, calc}
%\usetikzlibrary{shapes.geometric, arrows}
\usetikzlibrary{positioning, shapes.geometric, arrows}
\tikzstyle{block} = [rectangle, rounded corners, minimum width=2.2cm, minimum height=1cm,text centered, draw=black, fill=cyan!30]
\tikzstyle{arrow} = [thick,->,>=stealth, color=red]

% Define styles
\tikzstyle{block} = [rectangle, rounded corners, minimum width=1.4cm, minimum height=0.6cm, text centered, draw=black, fill=cyan!30]
\tikzstyle{arrow} = [thick, ->, >=stealth, color=blue]

\usepackage{listings}



\tikzstyle{process} = [rectangle, rounded corners, minimum width=3cm, minimum height=1cm, text centered, draw=black, fill=blue!20]
\tikzstyle{arrow} = [thick,->,>=stealth]

\tikzstyle{process} = [rectangle, rounded corners, minimum width=2.5cm, minimum height=1cm, text centered, draw=black, fill=blue!20, text width=2.5cm]
\tikzstyle{arrow} = [thick,->,>=stealth]


\tikzstyle{process} = [rectangle, rounded corners, minimum width=2.5cm, minimum height=1cm, text centered, draw=black, fill=blue!20, text width=2.5cm]
\tikzstyle{arrow} = [thick,->,>=stealth]


\lstset{
    basicstyle=\ttfamily\footnotesize,
    breaklines=true,
    language=VHDL,
    captionpos=b
}

\lstset{language=VHDL, 
        basicstyle=\ttfamily\footnotesize, 
        keywordstyle=\bfseries\color{blue},
        commentstyle=\color{purple},
        breaklines=true}



\lstset{
  language=Maple,
  basicstyle=\ttfamily\color{purple}, % All text is purple
  keywordstyle=\color{blue}\bfseries, % Keywords are blue and bold
  commentstyle=\color{green!70!black}, % Comments are dark green
  stringstyle=\color{red}, % Strings are red
  numbers=left, % Line numbers
  numberstyle=\tiny\color{gray}, % Line numbers in gray
  stepnumber=1,
  numbersep=8pt,
  showstringspaces=false, % Hide spaces in strings
  breaklines=true, % Break lines automatically
  frame=single, % Draw a box around the code
  captionpos=b, % Caption below the code
}


\usepackage{datetime}
\usepackage{ifthen}
\usepackage{siunitx}
\usepackage{alphabeta}



\usepackage{enumitem}
\newlist{indenteddesc}{description}{1}
\setlist[indenteddesc]{
  leftmargin=5em,  % labelindent+labelwidth+labelsep
  rightmargin=3em,
  labelindent=3em, % set equal to rightmargin
  labelwidth=1.5em,% choose a width the labels fit in
  labelsep=.5em
}

\PassOptionsToPackage{numbers}{natbib}
\usepackage{natbib}

\setlist[itemize]{itemsep=-5pt, topsep=0pt}

% Useful packages
% \usepackage[colorlinks=true, allcolors=blue]{hyperref}
\usepackage{hyperref} % Για ενεργά URLs
% \usepackage[hyphens]{url}

\title{ Extracted Content — Fluid Antenna Systems Enabling 6G (arXiv:2412.03839v1)}
\author{Georgios Giannakopoulos}
\date{January 2026}


% \section{Introduction}

% \section{COMPREHENSIVE TECHNICAL EXTRACTION}
% \noindent
% \textbf{Paper Title:} Extracted Content — Fluid Antenna Systems Enabling 6G (arXiv:2412.03839v1)

% \noindent
% \textbf{Authors:} Muhammad Asfar Saeed \& Augustine O. Nwajana

% \noindent
% \textbf{Institution:} University of Greenwich, Faculty of Engineering, Medway, United Kingdom

% \noindent
% \textbf{Conference:} 4th International Conference on Electrical, Computer and Energy Technologies (ICECET 2024)
% \noindent
% \textbf{Date:} 25-27 July 2024, Sydney, Australia
\begin{document}

\maketitle



Source file: 2412.03839v1.pdf\\
\section{Diagrams, figures, and illustrations}
Fig. 1: Examples of possible FAS structures. (page 2)\\
 
Fig. 1: Examples of possible FAS structures.\\

\section{Context (verbatim excerpt)}
A. Structures An obvious design for a FA is liquid metal (LM) fiber [4], as depicted in Fig. 1(a). It is made by injecting liquid metal into elastic micro-channels or hollow fibers. LM fibers, made by molding silicone or commercially available hollow fibers, offer excellent mechanical elasticity and can withstand numerous strain cycles without degradation. The high conductivity of liquid metal minimizes resistive losses, thereby reducing energy consumption during signal transmission. This durable and lightweight design ensures long-term reliability in harsh environments while maintaining energy efficiency. Another design for a FA is to utilize conductive fluid inside a tube-like container, in which the fluid can be mobilized by a digitally controlled nano-pump [2, Section VI-B], as illustrated in Fig. 1(b). The controller adjusts dynamically the antenna’s position (a.k.a. port), shape and size based on the channel condition in real time, ensuring best performance. This agility makes it ideal for dynamic environments, or rapidly changing network settings as typically faced at mobile devices, obtaining consistency and stability in link connections. Fig. 1(c) illustrates another examp

Fig. 2: Potential applications of FAS in wireless communication systems. (page 5)
 
Fig. 2: Potential applications of FAS in wireless communication systems.


\section{Context (verbatim excerpt)}
III. APPLICATIONS ASSISTED BY FAS With the reconfigurability FAS possesses, it is expected that FAS can find many applications in wireless networks. In this section, we discuss several obvious applications, as illustrated in Fig. 2, that FAS can contribute to massively. A. Simultaneous Wireless Information, and Power Transfer (SWIPT) SWIPT has been an attractive feature to support the growing number of battery-powered devices for wireless connections. The fact that SWIPT aims to deliver data and energy concurrently makes it a smart technique but it also puts strains on the SWIPT transmitter. If the transmitter relies on a FPA, then both communication and power transfer efficiencies will be degraded. In this case, replacing FPA by FAS can empower the SWIPT transmitter for more DoF to elevate the performance of both communication and power transfer [9]. This is especially important if the transmitter is a mobile device as the number of RF chains tends to be very small. Evidently, a SWIPT mobile receiver benefits as well if FAS is deployed. A FPA has no DoF to optimize the received signal for SWIPT but a FAS can reconfigure itself to receive the best signal for power transfer and anot
 Fig. 3: $Pmax/σ_I^2$ versus R. (page 7)
 
 Fig. 3: $Pmax/σ_I^2$ versus R.

\section{Context (verbatim excerpt)}
Consider a FAS-assisted SWIPT system, which contains a BS, an information receiver (IR), and an energy receiver -5 0 5 10 15 0 1 2 3 4 5 6 7 8 Fig. 3: Pmax/σ2 I versus R. (ER). The BS is equipped with 4 FAs. The IR and ER are both equipped with a single FA. We assume that the energy harvesting efficiency is set to 0.5 (i.e., 50\%). The minimum distance between two FAs is set to be D = λ/2 with λ = 1 m. All other parameters are the same as those used in [9]. In this case study, we adopt the field response-based channel model to capture the geometric relationships between the FAs. This modeling approach considers finite multipath propagation effects (3 channel paths were considered in the simulations) and the angles of signal departure and arrival among different antenna positions, allowing for the optimization of antenna positions to enhance communication performance. Explicitly, in [9], an iterative algorithm was proposed for jointly optimizing the transmit beamforming of BS, the locations of receive FAs within IR and ER, and the locations of transmit FA within the BS for maximizing the communication rate at the IR, denoted by R. We compare this scheme (called ‘FAS’) to the following three benchmarks: 1) Transmit FA (TFA): The BS has M FAs, each of which is equipped with an independent RF chain, while the IR and ER are both equipped with a single FPA. 2) Receive FA (RFA): The BS
Fig. 4: Outage probability versus the number of ports N. (page 8)
 
Fig. 4: Outage probability versus the number of ports N.
Context (verbatim excerpt)
Another direction is combining FAS and RIS systems. This case study considers a RIS-assisted downlink communication system comprising a BS with a single FPA, a RIS with M reflecting elements, and a mobile user (MU) equipped with a single FA. Also, the direct link between the BS and the MU is assumed to be broken by obstacles. The distance between the BS and RIS and that between the RIS and the MU are both set to 200 meters. The wireless channel is assumed to undergo rich scattering, hence following Rayleigh fading. Detailed setup of other system parameters is set according to [11]. Results in Fig. 4 are provided for the outage probability against the number of ports, N, of FAS, where the value of M is set to 40 or 45. The noise power is $σ_n^2 = 10-8 W$, and the transmit power is $PS = 0.1 W$. Therefore, the average received SNR is $E(PS|γk|2/σ2) = 10 dB$ and 11.0231 dB for M = 40 and M = 45, respectively. The target data rate is R = 3 bit/s/Hz. In the numerical results, several different channel models with different properties were used. The block correlation model in [11] is denoted by ‘CLT-BC’. This model is known to have great accuracy and analytical tractability. In addition, we als

\section{Tables and tabular data}




Table I: FAS characteristics and their contributions/advantages to 6G systems (page 3)

\begin{table}[]
\caption{Table I: FAS characteristics and their contributions/advantages to 6G systems (page 3)}
\label{tab:my-table}
\begin{tabular}{|l|l|l|l|}
\hline
Characteristic type &
  Subcategory &
  Application example &
  Key contributions/advantages \\ \hline
Material &
  LM (M1) &
  LM Fiber; conductive fluid with controller &
  High flexibility; efficient signal transmission; real-time   adaptability. \\ \hline
Material &
  Non-metallic liquids (M2) &
  Water antenna &
  Environmentally friendly design; tuneable resonance for   energy-efficient systems. \\ \hline
Material &
  Metallic pixels (M3) &
  PRA &
  Rapid electronic reconfiguration; microsecond-level switching or   faster for IoT and high-speed networks. \\ \hline
Shape &
  Filament (S1) &
  Stretchable clothes &
  Lightweight and flexible; suitable for wearables; easily   integrated into IoT. \\ \hline
Shape &
  Planar (S2) &
  Metallophobic surface; PRA &
  Flat structure for easy integration; suitable for beamforming and   efficient coverage. \\ \hline
Shape &
  3D structure (S3) &
  Stacking 3D LM; water antenna &
  Enhanced volumetric efficiency and broader coverage; optimal for   antenna deployment in dense areas. \\ \hline
Dynamic characteristics control &
  Controllable liquid flow (C1) &
  Conductive fluid with controller &
  Real-time optimization for changing environments; particularly   suitable for mobile devices. \\ \hline
Dynamic characteristics control &
  Pattern-controlled liquid (C2) &
  Metallophobic surface &
  Precision control using patterns; ideal for forming complex   conductive paths. \\ \hline
Dynamic characteristics control &
  Amount-controlled liquid (C3) &
  Stacking 3D LM &
  Fine-tuning performance through liquid quantity adjustment;   offering adaptable solutions. \\ \hline
Dynamic characteristics control &
  Electronic switching control (C4) &
  PRA &
  Microsecond-level reconfiguration via RF switches; suitable for   rapid adaptation in dynamic communication environments. \\ \hline
Channel model &
  Field response-based channel (F1) &
  Finite-scattering environments (typical for millimeter-wave   communications) &
  (1) Use FAS as an additional DoF to empower MIMO for jointly   optimizing antenna positioning, orientation, polarization and beamforming;   and (2) exploit fine spatial-domain resolution to activate received signal at   interference null for multiple access. \\ \hline
Channel model &
  Correlation-based channel (F2) &
  Rich-scattering environments &
  Statistical analysis of port correlation (e.g., Jake’s model;   copula-based analysis) for achievable performance. \\ \hline
\end{tabular}
\end{table}


% Characteristic type	Subcategory	Application example	Key contributions/advantages
% Material	LM (M1)	LM Fiber; conductive fluid with controller	High flexibility; efficient signal transmission; real-time adaptability.
% Material	Non-metallic liquids (M2)	Water antenna	Environmentally friendly design; tuneable resonance for energy-efficient systems.
% Material	Metallic pixels (M3)	PRA	Rapid electronic reconfiguration; microsecond-level switching or faster for IoT and high-speed networks.
% Shape	Filament (S1)	Stretchable clothes	Lightweight and flexible; suitable for wearables; easily integrated into IoT.
% Shape	Planar (S2)	Metallophobic surface; PRA	Flat structure for easy integration; suitable for beamforming and efficient coverage.
% Shape	3D structure (S3)	Stacking 3D LM; water antenna	Enhanced volumetric efficiency and broader coverage; optimal for antenna deployment in dense areas.
% Dynamic characteristics control	Controllable liquid flow (C1)	Conductive fluid with controller	Real-time optimization for changing environments; particularly suitable for mobile devices.
% Dynamic characteristics control	Pattern-controlled liquid (C2)	Metallophobic surface	Precision control using patterns; ideal for forming complex conductive paths.
% Dynamic characteristics control	Amount-controlled liquid (C3)	Stacking 3D LM	Fine-tuning performance through liquid quantity adjustment; offering adaptable solutions.
% Dynamic characteristics control	Electronic switching control (C4)	PRA	Microsecond-level reconfiguration via RF switches; suitable for rapid adaptation in dynamic communication environments.
% Channel model	Field response-based channel (F1)	Finite-scattering environments (typical for millimeter-wave communications)	(1) Use FAS as an additional DoF to empower MIMO for jointly optimizing antenna positioning, orientation, polarization and beamforming; and (2) exploit fine spatial-domain resolution to activate received signal at interference null for multiple access.
% Channel model	Correlation-based channel (F2)	Rich-scattering environments	Statistical analysis of port correlation (e.g., Jake’s model; copula-based analysis) for achievable performance.


\section{Simulation results and performance metrics}
\section{Case study A: FAS-SWIPT systems (page 7)}
Consider a FAS-assisted SWIPT system, which contains a BS, an information receiver (IR), and an energy receiver -5 0 5 10 15 0 1 2 3 4 5 6 7 8 Fig. 3: Pmax/σ2 I versus R. (ER). The BS is equipped with 4 FAs. The IR and ER are both equipped with a single FA. We assume that the energy harvesting efficiency is set to 0.5 (i.e., 50\%). The minimum distance between two FAs is set to be D = λ/2 with λ = 1 m. All other parameters are the same as those used in [9]. In this case study, we adopt the field response-based channel model to capture the geometric relationships between the FAs. This modeling approach considers finite multipath propagation effects (3 channel paths were considered in the simulations) and the angles of signal departure and arrival among different antenna positions, allowing for the optimization of antenna positions to enhance communication performance. Explicitly, in [9], an iterative algorithm was proposed for jointly optimizing the transmit beamforming of BS, the locations of receive FAs within IR and ER, and the locations of transmit FA within the BS for maximizing the communication rate at the IR, denoted by R. We compare this scheme (called ‘FAS’) to the following three benchmarks: 1) Transmit FA (TFA): The BS has M FAs, each of which is equipped with an independent RF chain, while the IR and ER are both equipped with a single FPA. 2) Receive FA (RFA): The BS has M FPAs, while the IR and ER are both equipped with a single FA each. 3) FPA: The BS is equipped with M FPAs, while the IR and ER are both equipped with a single FPA. Our interest is to analyze the effect of the transmit power to noise ratio, denoted by $Pmax/σ2 I$, with M = 4 as shown in Fig. 3. The results in this figure show that the communication rate for all the schemes increases with Pmax/σ2 I. This is expected as a high SNR should allow a high data rate to be achieved and increase harvested energy. Additionally, the ‘FAS’ scheme consistently outperforms the benchmarks, demonstrating its effectiveness in optimizing the communication rate. The substantial improvements, particularly the 73.1\% enhancement over the FPA scheme, highlight the superior gain provided by FAS in optimizing the system performance.

\section{Case study B: FAS-RIS systems (page 8)}
Another direction is combining FAS and RIS systems. This case study considers a RIS-assisted downlink communication system comprising a BS with a single FPA, a RIS with M reflecting elements, and a mobile user (MU) equipped with a single FA. Also, the direct link between the BS and the MU is assumed to be broken by obstacles. The distance between the BS and RIS and that between the RIS and the MU are both set to 200 meters. The wireless channel is assumed to undergo rich scattering, hence following Rayleigh fading. Detailed setup of other system parameters is set according to [11]. Results in Fig. 4 are provided for the outage probability against the number of ports, N, of FAS, where the value of M is set to 40 or 45. The noise power is $σ_n^2 = 10-8 W$, and the transmit power is $PS = 0.1 W$. Therefore, the average received SNR is $E(PS|γk|2/σ2) = 10 dB$ and $11.0231 dB$ for M = 40 and M = 45, respectively. The target data rate is R = 3 bit/s/Hz. In the numerical results, several different channel models with different properties were used. The block correlation model in [11] is denoted by ‘CLT-BC’. This model is known to have great accuracy and analytical tractability. In addition, we also have the constant correlation model, denoted by ‘Constant’. This model is preferred when simplicity is the priority but it tends to overestimate the performance. As observed, increasing the number of FAS ports N from 5 to 50 significantly reduces outage probabilities, emphasizing the critical role of FAS in enhancing communication performance. Similarly, increasing the number of RIS elements M from 40 to 45 provides additional improvements, highlighting the potential of RIS technology to enhance network reliability. These results demonstrate a synergistic relationship between FAS and RIS, which can be effectively leveraged to significantly improve the reliability and efficiency of wireless networks. VI. CONCLUSIONS Reconfigurable antennas have begun to demonstrate their value in revolutionizing wireless communication systems, and fluid antenna systems (FAS) are taking the lead by showcasing unparalleled flexibility and efficiency. This article categorizes the diverse implementation techniques for realizing FAS, emphasizing their unique strengths and characteristics. We have also highlighted key applications where FAS stands out, such as SWIPT, ISAC, NOMA, RIS, PLS, MEC, and others, illustrating the unique advantages it brings to existing technologies. Furthermore, we addressed pressing challenges surrounding FAS, identifying five strategic research directions that can guide future innovations. To solidify its potential, the article concluded with two case studies that illustrate the clear benefits and transformative potential of FAS in advancing next-generation wireless systems. These findings underscore the pivotal role of FAS in shaping the future of wireless communications and provide a roadmap for further exploration and development in this exciting field. 


REFERENCES
[1] K. K. Wong, A. Shojaeifard, K. F. Tong, and Y. Zhang, “Fluid antenna system,” IEEE Trans. Wirel. Commun., vol. 20, no. 3, pp. 1950–1962, Mar. 2021. [2] W. K. New et al., “A tutorial on fluid antenna system for 6G networks: Encompassing communication theory, optimization methods and hardware designs,” IEEE Commun. Surv. \& Tut., doi:10.1109/COMST.2024. 3498855, 2024. [3] L. Zhu, W. Ma, B. Ning and R. Zhang, “Movable-antenna enhanced multiuser communication via antenna position optimization,” IEEE Trans. Wireless Commun., vol. 23, no. 7, pp. 7214–7229, Jul. 2024. [4] J. Ma et al., “Shaping a soft future: Patterning liquid metals,” Adv. Mater., vol. 35, no. 19, May 2023. [5] I. D. Joshipura et al., “Are contact angle measurements useful for oxidecoated liquid metals?,” Langmuir, vol. 37, pp. 10914–10923, Sept. 2021. [6] C. Borda-Fortuny, K. F. Tong, A. Al-Armaghany and K. K. Wong, “A low-cost fluid switch for frequency-reconfigurable Vivaldi antenna,” IEEE Antennas \& Wireless Propag. Lett., vol. 16, pp. 3151–3154, 2017. [7] R. Lin et al., “Digitally-embroidered liquid metal electronic textiles for wearable wireless systems,” Nat. Commun., vol. 13, no. 1, p. 2190, Apr. 2022. [8] J. Zhang et al., “A novel pixel-based reconfigurable antenna applied in fluid antenna systems with high switching speed,” IEEE Open J. Antennas \& Propag., doi:10.1109/OJAP.2024.3489215, 2024. [9] L. Zhou et al., “Fluid antenna-assisted simultaneous wireless information and power transfer systems,” arXiv preprint, arXiv:2407.11307v2, Jul. 2024. [10] C. Wang et al., “Fluid antenna system liberating multiuser MIMO for ISAC via deep reinforcement learning,” IEEE Trans. Wirel. Commun., vol. 23, no. 9, pp. 10879–10894, Sept. 2024. [11] X. Lai et al., “FAS-RIS: A block-correlation model analysis,” IEEE Trans. Veh. Technol., doi:10.1109/TVT.2024.3480234, 2024. [12] F. R. Ghadi et al., “Physical layer security over fluid antenna systems: Secrecy performance analysis,” IEEE Trans. Wirel. Commun., doi:10. 1109/TWC.2024.3463488, 2024. [13] Y. Zuo et al., “Fluid antenna for mobile edge computing,” IEEE Commun. Lett., vol. 28, no. 7, pp. 1728–1732, Jul. 2024. [14] H. Xu et al., “Channel estimation for FAS-assisted multiuser mmWave systems,” IEEE Commun. Lett., vol. 23, no. 3, pp. 632–636, Mar. 2024. [15] C. Wang et al., “AI-empowered fluid antenna systems: Opportunities, challenges, and future directions,” IEEE Wirel. Commun., vol. 31, no. 5, pp. 34–41, Oct. 2024.

\section{Technical specifications of each antenna or relevant component}
\begin{itemize}
    \item \textbf{FAS structures and implementation examples (Section II-A / Fig. 1):} LM fiber; conductive fluid with controller (nano-pump, digital controller); metallophobic surface; stacking 3D LM (direct-write printing); water antenna; stretchable clothes with embedded LM antennas; pixel-based reconfigurable antenna (PRA).
    \item \textbf{PRA prototype detail: The paper notes a FAS prototype with 12 states} developed in recent work [8] (mentioned in Section II-A).
    \item \textbf{Case study A (FAS-SWIPT) parameter values (Section V-A):} BS equipped with 4 fluid antennas; IR and ER each equipped with a single fluid antenna; energy harvesting efficiency 0.5 (50\%); minimum distance between two FAs D = λ/2 with λ = 1 m; field-response channel model with 3 channel paths considered.
    \item Case study B (FAS-RIS) parameter values (Section V-B): BS has a single fixed-position antenna; RIS has M reflecting elements with M = 40 or 45; direct BS–MU link blocked; BS–RIS distance 200 m and RIS–MU distance 200 m; Rayleigh fading (rich scattering); noise power $σ_n^2 = 10-8 W$; transmit power $P_S = 0.1 W$; target rate $R = 3 bit/s/Hz$; reported average received SNR 10 dB (M=40) and 11.0231 dB (M=45).
\end{itemize}

\section{Design techniques, configurations, and methodologies}
    \begin{itemize}
        \item \textbf{FAS characterization dimensions (Section II):} Material types (M1–M3), shape types (S1–S3), dynamic control types (C1–C4), and channel modeling types (F1–F2).
        \item \textbf{Dynamic control approaches (Section II-D):} Controllable liquid flow (nano-pump), pattern-controlled liquid (selective adhesion/patterning), amount-controlled liquid (varying quantity), and electronic switching control (RF switches in PRA).
        \item \textbf{Channel modeling approaches (Section II-E):} Field response-based modeling for finite scattering; correlation-based modeling (e.g., Jake’s model) for rich scattering; notes that models need upgrades to capture shape/polarization and state-based (pixel) reconfiguration.
        \item \textbf{Optimization in case study A (Section V-A):} Iterative algorithm jointly optimizing BS transmit beamforming, receive FA locations (IR/ER), and transmit FA locations (BS) to maximize IR rate R, compared against TFA/RFA/FPA benchmarks.
        \item \textbf{Analysis models in case study B (Section V-B):} Block correlation model (‘CLT-BC’) and constant correlation model used for outage analysis vs. number of ports N.
        \item \textbf{Research directions (Section IV):} Channel estimation (compressed sensing / AI); versatile channel modeling; robust beamforming and FA location optimization under imperfect CSI; localization; AI-driven FAS.
    \end{itemize}

\section{Citations and references}

\begin{enumerate}
    \item [1] K. K. Wong, A. Shojaeifard, K. F. Tong, and Y. Zhang, “Fluid antenna system,” IEEE Trans. Wirel. Commun., vol. 20, no. 3, pp. 1950–1962, Mar. 2021.
    \item [2] W. K. New et al., “A tutorial on fluid antenna system for 6G networks: Encompassing communication theory, optimization methods and hardware designs,” IEEE Commun. Surv. \& Tut., doi:10.1109/COMST.2024. 3498855, 2024.
    \item [3] L. Zhu, W. Ma, B. Ning and R. Zhang, “Movable-antenna enhanced multiuser communication via antenna position optimization,” IEEE Trans. Wireless Commun., vol. 23, no. 7, pp. 7214–7229, Jul. 2024.
    \item [4] J. Ma et al., “Shaping a soft future: Patterning liquid metals,” Adv. Mater., vol. 35, no. 19, May 2023.
    \item [5] I. D. Joshipura et al., “Are contact angle measurements useful for oxidecoated liquid metals?,” Langmuir, vol. 37, pp. 10914–10923, Sept. 2021.
    \item [6] C. Borda-Fortuny, K. F. Tong, A. Al-Armaghany and K. K. Wong, “A low-cost fluid switch for frequency-reconfigurable Vivaldi antenna,” IEEE Antennas \& Wireless Propag. Lett., vol. 16, pp. 3151–3154, 2017.
    \item [7] R. Lin et al., “Digitally-embroidered liquid metal electronic textiles for wearable wireless systems,” Nat. Commun., vol. 13, no. 1, p. 2190, Apr. 2022.
    \item [8] J. Zhang et al., “A novel pixel-based reconfigurable antenna applied in fluid antenna systems with high switching speed,” IEEE Open J. Antennas \& Propag., doi:10.1109/OJAP.2024.3489215, 2024.
    \item [9] L. Zhou et al., “Fluid antenna-assisted simultaneous wireless information and power transfer systems,” arXiv preprint, arXiv:2407.11307v2, Jul. 2024.
    \item [10] C. Wang et al., “Fluid antenna system liberating multiuser MIMO for ISAC via deep reinforcement learning,” IEEE Trans. Wirel. Commun., vol. 23, no. 9, pp. 10879–10894, Sept. 2024.
    \item [11] X. Lai et al., “FAS-RIS: A block-correlation model analysis,” IEEE Trans. Veh. Technol., doi:10.1109/TVT.2024.3480234, 2024.
    \item [12] F. R. Ghadi et al., “Physical layer security over fluid antenna systems: Secrecy performance analysis,” IEEE Trans. Wirel. Commun., doi:10. 1109/TWC.2024.3463488, 2024.
    \item [13] Y. Zuo et al., “Fluid antenna for mobile edge computing,” IEEE Commun. Lett., vol. 28, no. 7, pp. 1728–1732, Jul. 2024.
    \item [14] H. Xu et al., “Channel estimation for FAS-assisted multiuser mmWave systems,” IEEE Commun. Lett., vol. 23, no. 3, pp. 632–636, Mar. 2024.
    \item [15] C. Wang et al., “AI-empowered fluid antenna systems: Opportunities, challenges, and future directions,” IEEE Wirel. Commun., vol. 31, no. 5, pp. 34–41, Oct. 2024.
\end{enumerate}

\end{document}

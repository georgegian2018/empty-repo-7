%%%%%%%%%%%%%%%%%%%%%%%%%%%%%%%%%%%%%%%%%%%%%%%%%%%%%%%%%%%%%%%%%%%%%%%%%%%%%%%%%%%%
%%----------------------------------------------------------------------------------
% DO NOT Change this is the required setting A4 page, 11pt, onside print, book style
%%----------------------------------------------------------------------------------
\documentclass[a4paper,11pt,oneside]{article} 
%\usepackage{CS_report} % DO NOT REMOVE THIS LINE. 
%%%%%%%%%%%%%%%%%%%%%%%%%%%%%%%%%%%%%%%%%%%%%%%%%%%%%%%%%%%%%%%%%%%%%%%%%%%%%%%%%%%%
\usepackage[english]{babel}
% \usepackage{algorithm}
% \usepackage{algorithmic}
\usepackage{dirtytalk}
\usepackage[utf8]{inputenc}
\usepackage{graphicx}
\usepackage{amssymb,amsmath,amsthm,amsfonts}
\usepackage{multicol}
\usepackage{multirow} % used for tables to merge multiple rows
% \usepackage[table]{xcolor}
\usepackage{float}

\usepackage{lettrine,paralist,fancyhdr}
\usepackage{nomencl,makeidx,pdfpages}
\usepackage{setspace}

\usepackage{longtable}
\usepackage{booktabs} % used for tables

\usepackage{bigdelim} % used for tables to set spacing
\usepackage{bigstrut} % used for tables to set spacing
\usepackage{tabularray}
\usepackage{tabularx}
\usepackage{ragged2e}

% Set page size and margins
% Replace `letterpaper' with `a4paper' for UK/EU standard size
% \usepackage[letterpaper,top=2cm,bottom=2cm,left=3cm,right=3cm,marginparwidth=1.75cm]{geometry}

\usepackage{float}
 \let\Bbbk\relax

%\DeclareUnicodeCharacter{2212}{-} %It's for the negative sign in exponent to be shown
\usepackage[table,xcdraw]{xcolor}  


\usepackage[labelfont=bf]{caption}
\usepackage{subcaption}
\captionsetup[subfigure]{labelformat=simple, labelsep=colon}
\renewcommand{\thesubfigure}{(\alph{subfigure})}
\usepackage{epstopdf}
%\usepackage{comment}
\usepackage{textcase}

%%%%%%%%%%%%%%%%%%%%%%%%%%%%%%%%%%%%%%%%%%%%%%%%%%%%%%%%%%%%
\usepackage{listings}
\usepackage{xcolor}
\lstset{
  basicstyle=\footnotesize\ttfamily,  % Adjust font size for fitting code
  keywordstyle=\color{blue},
  commentstyle=\color{gray},
  stringstyle=\color{red},
  numbers=left,
  numberstyle=\tiny,
  stepnumber=1,
  breaklines=true,     % Automatically break long lines
  tabsize=4,
  captionpos=b,
  frame=single
}

\lstset{
  language=VHDL,
  basicstyle=\ttfamily\footnotesize,
  keywordstyle=\bfseries\color{blue},
  commentstyle=\color{purple},
  breaklines=true
}

\lstset{
    basicstyle=\ttfamily, % Monospaced font for the code
    frame=single, % Single line border around the listing
    xleftmargin=0.5cm, % Adjust the left margin (resize box from the left)
    xrightmargin=0.5cm, % Adjust the right margin (resize box from the right)
    aboveskip=0.5cm, % Add space above the listing
    belowskip=0.5cm, % Add space below the listing
    breaklines=true % Enable line breaking for long lines
}

\lstset{language=Matlab, 
    basicstyle=\ttfamily\footnotesize, 
    keywordstyle=\color{blue}\bfseries, 
    commentstyle=\color{purple},  % Comments in purple
    stringstyle=\color{brown}, 
    numbers=left, 
    numberstyle=\tiny\color{gray}, 
    stepnumber=1, 
    numbersep=10pt, 
    frame=single, 
    breaklines=true, 
    breakatwhitespace=false, 
    showspaces=false, 
    showtabs=false, 
    tabsize=2,
    captionpos=b
}






% Define Maple language with MATLAB-style aesthetics
\lstdefinelanguage{Maple}{
    morecomment=[l]{\#},           % Line comments start with #
    morestring=[b]'',               % Strings are enclosed in double quotes
    sensitive=false                % Case-insensitive
}

% lstset configuration for MAPLE
\lstset{
    language=Maple,                       % Use the custom Maple language
    basicstyle=\ttfamily\footnotesize,    % Monospace font, small size
    commentstyle=\color{red},             % Comments in red
    stringstyle=\color{brown},            % Strings in brown
    numbers=left,                         % Line numbers on the left
    numberstyle=\tiny\color{gray},        % Line number style
    stepnumber=1,                         % Increment of line numbers
    numbersep=10pt,                       % Distance from line numbers to code
    frame=single,                         % Frame around the code block
    breaklines=true,                      % Break long lines automatically
    breakatwhitespace=false,              % Do not restrict breaks to spaces
    postbreak=\mbox{\textcolor{red}{$\hookrightarrow$}\space}, % Continuation marker
    showspaces=false,                     % Do not show spaces explicitly
    showtabs=false,                       % Do not show tabs explicitly
    tabsize=4,                            % Tab size equivalent to 4 spaces
    captionpos=b,                         % Caption position at the bottom
    xleftmargin=10pt,                     % Left margin for the code block
    framexleftmargin=5pt                  % Inner left margin of the frame
}



%%%%%%%%%%%%%%%%%%%%%%%%%%%%%%%%%%%%%%%%%%%%%%%%%%%%%%%%%%%%

\usepackage{pgfplots}
\usepackage{pgfplotstable}
\usepackage{tikz}
% Enable compatibility mode (required to avoid errors)
\pgfplotsset{compat=newest}
\pgfplotsset{compat=1.18}
\usetikzlibrary{3d}
\usepackage{eqparbox}

\usepackage{tikz}
\usepackage{tikz-3dplot}
\usetikzlibrary{positioning}
\usetikzlibrary{shapes.geometric, calc}
\usetikzlibrary{shapes.geometric, arrows}
\usetikzlibrary{shapes.geometric, arrows.meta}

%\usetikzlibrary{positioning}
%\usetikzlibrary{shapes.geometric, arrows, calc}
%\usetikzlibrary{shapes.geometric, arrows}
\usetikzlibrary{positioning, shapes.geometric, arrows}
\tikzstyle{block} = [rectangle, rounded corners, minimum width=2.2cm, minimum height=1cm,text centered, draw=black, fill=cyan!30]
\tikzstyle{arrow} = [thick,->,>=stealth, color=red]

% Define styles
\tikzstyle{block} = [rectangle, rounded corners, minimum width=1.4cm, minimum height=0.6cm, text centered, draw=black, fill=cyan!30]
\tikzstyle{arrow} = [thick, ->, >=stealth, color=blue]

\usepackage{listings}



\tikzstyle{process} = [rectangle, rounded corners, minimum width=3cm, minimum height=1cm, text centered, draw=black, fill=blue!20]
\tikzstyle{arrow} = [thick,->,>=stealth]

\tikzstyle{process} = [rectangle, rounded corners, minimum width=2.5cm, minimum height=1cm, text centered, draw=black, fill=blue!20, text width=2.5cm]
\tikzstyle{arrow} = [thick,->,>=stealth]


\tikzstyle{process} = [rectangle, rounded corners, minimum width=2.5cm, minimum height=1cm, text centered, draw=black, fill=blue!20, text width=2.5cm]
\tikzstyle{arrow} = [thick,->,>=stealth]


\lstset{
    basicstyle=\ttfamily\footnotesize,
    breaklines=true,
    language=VHDL,
    captionpos=b
}

\lstset{language=VHDL, 
        basicstyle=\ttfamily\footnotesize, 
        keywordstyle=\bfseries\color{blue},
        commentstyle=\color{purple},
        breaklines=true}



\lstset{
  language=Maple,
  basicstyle=\ttfamily\color{purple}, % All text is purple
  keywordstyle=\color{blue}\bfseries, % Keywords are blue and bold
  commentstyle=\color{green!70!black}, % Comments are dark green
  stringstyle=\color{red}, % Strings are red
  numbers=left, % Line numbers
  numberstyle=\tiny\color{gray}, % Line numbers in gray
  stepnumber=1,
  numbersep=8pt,
  showstringspaces=false, % Hide spaces in strings
  breaklines=true, % Break lines automatically
  frame=single, % Draw a box around the code
  captionpos=b, % Caption below the code
}


\usepackage{datetime}
\usepackage{ifthen}
\usepackage{siunitx}
\usepackage{alphabeta}



\usepackage{enumitem}
\newlist{indenteddesc}{description}{1}
\setlist[indenteddesc]{
  leftmargin=5em,  % labelindent+labelwidth+labelsep
  rightmargin=3em,
  labelindent=3em, % set equal to rightmargin
  labelwidth=1.5em,% choose a width the labels fit in
  labelsep=.5em
}

\PassOptionsToPackage{numbers}{natbib}
\usepackage{natbib}

\setlist[itemize]{itemsep=-5pt, topsep=0pt}

% Useful packages
% \usepackage[colorlinks=true, allcolors=blue]{hyperref}
\usepackage{hyperref} % Για ενεργά URLs
% \usepackage[hyphens]{url}

\title{A Tutorial on Fluid Antenna System for 6G Networks}
\author{Georgios Giannakopoulos}
\date{January 2026}

\begin{document}

\maketitle

 
\section*{Comprehensive Research Extraction and Organization}

\noindent
\textbf{Authors:} Wee Kiat New, Kai-Kit Wong, Hao Xu, Chao Wang, Farshad Rostami Ghadi, Jichen Zhang, Junhui Rao, Ross Murch, Pablo Ramírez-Espinosa, David Morales-Jimenez, Chan-Byoung Chae, Kin-Fai Tong \\

\noindent

\textbf{Publication:} IEEE Tutorial Paper on Fluid Antenna System (FAS)\\
\noindent

\textbf{Citation:} New, W.K.; Wong, K.K.; Xu, H.; et al. A Tutorial on Fluid Antenna System for 6G Networks: Encompassing Communication Theory, Optimization Methods and Hardware Designs.\\

\newpage

\section*{TABLE OF CONTENTS}
\noindent

\begin{enumerate}
    \item Document Information
    \item Section 1: Diagrams and Figures
    \item Section 2: Tables and Tabular Data
    \item Section 3: Equations and Mathematical Models
    \item Section 4: Channel Models
    \item Section 5: Channel Estimation Methods
    \item Section 6: Hardware Designs
    \item Section 7: Technical Specifications
    \item Section 8: Performance Results and Metrics
    \item Section 9: Design Methodologies
    \item Section 10: Comprehensive References
\end{enumerate}

\newpage

\section*{DOCUMENT INFORMATION}

% | Detail | Information |
% |--------|-----------|
% | Document Type | IEEE Tutorial Paper |
% | Subject | Fluid Antenna Systems for 6G Networks |
% | Focus Areas | Communication Theory, Optimization Methods, Hardware Designs |
% | Scope | Channel Modeling, Signal Processing, Information Theory, Multiple Access, Hardware Implementation |
% | Primary Audience | Researchers in 6G, Communications Engineering, Antenna Design |

% Please add the following required packages to your document preamble:
% \usepackage[table,xcdraw]{xcolor}
% Beamer presentation requires \usepackage{colortbl} instead of \usepackage[table,xcdraw]{xcolor}
\begin{table}[ht!]
\caption{}
\label{tab:my-table}
\begin{tabular}{ll}
\rowcolor[HTML]{FFFFFF} 
\multicolumn{1}{c}{\cellcolor[HTML]{FFFFFF}{\color[HTML]{1F2328} \textbf{Detail}}} &
  \multicolumn{1}{c}{\cellcolor[HTML]{FFFFFF}{\color[HTML]{1F2328} \textbf{Information}}} \\
\rowcolor[HTML]{FFFFFF} 
{\color[HTML]{1F2328} \textbf{Document Type}}    & {\color[HTML]{1F2328} IEEE Tutorial Paper}                                           \\
\rowcolor[HTML]{F6F8FA} 
{\color[HTML]{1F2328} \textbf{Subject}}          & {\color[HTML]{1F2328} Fluid Antenna Systems for 6G Networks}                         \\
\rowcolor[HTML]{FFFFFF} 
{\color[HTML]{1F2328} \textbf{Focus Areas}}      & {\color[HTML]{1F2328} Communication Theory, Optimization Methods, Hardware Designs}  \\
\rowcolor[HTML]{F6F8FA} 
{\color[HTML]{1F2328} \textbf{Scope}} &
  {\color[HTML]{1F2328} Channel Modeling, Signal Processing, Information Theory, Multiple Access, Hardware Implementation} \\
\rowcolor[HTML]{FFFFFF} 
{\color[HTML]{1F2328} \textbf{Primary Audience}} & {\color[HTML]{1F2328} Researchers in 6G, Communications Engineering, Antenna Design}
\end{tabular}
\end{table}



\subsection*{Key Research Institutions}
\begin{itemize}
    \item University College London, Department of Electronic and Electrical Engineering
    \item Yonsei University, Yonsei Frontier Laboratory
    \item Xidian University, Integrated Service Networks Lab
    \item Hong Kong University of Science and Technology, Department of Electronic and Computer Engineering
    \item Universidad de Granada, Department of Signal Theory, Networking and Communications
\end{itemize}



\section*{SECTION 1: DIAGRAMS AND FIGURES}

\subsection*{Figure 1: IMT-2030 Usage Scenarios for 6G}
\textbf{Context:} Illustrates the six anticipated usage scenarios for 6G networks as defined by the International Telecommunication Union (ITU-T):
\begin{itemize}
    \item Immersive communication
    \item Hyper reliable, low-latency communication (HRLLC)
    \item Massive communication
    \item Artificial intelligence (AI) and communication
    \item Ubiquitous connectivity
    \item Integrated sensing and communication (ISAC)
\end{itemize}

\subsection*{Figure 2: Organization of the Tutorial}
\textbf{Context:} Provides a roadmap showing the structure of the tutorial paper with interconnections between different sections covering channel models, estimation methods, fundamentals, multiple access techniques, hardware designs, standardization, and future challenges.

\subsection*{Figure 3: A Schematic of 1D Fluid Antenna Structure}
\textbf{Components:}
\begin{itemize}
    \item Linear arrangement of N preset port locations
    \item Ports uniformly distributed over a length of $W\lambda$
    \item Single active radiating element (active port)
    \item Port spacing determined by wavelength $\lambda$
\end{itemize}

\noindent
\textbf{Description:} Shows the basic architecture of a one-dimensional fluid antenna system with evenly spaced ports along a linear dimension.

\subsection*{Figure 4: Average Variance of Channel Coefficients vs Approximation Level}
\textbf{Parameters:} 
\begin{itemize}
    \item X-axis: Approximation level $\hat{N}$ (number of eigenvalues used)
    \item Y-axis: Average variance (normalized to 1)
    \item Variables: Different values of W (0.5, 1, 2, 3, 4, 5)
    \item Total ports: $N = 100$
\end{itemize}

\noindent
\textbf{Key Finding:} Shows convergence of channel variance approximation with relatively few eigenvalues, demonstrating that accurate approximation can be achieved with $\hat{N} \ll N$.


\subsection*{Figure 5: Example of 2D FAS Receiver}
\textbf{Illustration:} Shows mapping between 2D port indices and 1D indexing scheme, demonstrating how a 2D array of ports can be linearized for analysis.

\subsection*{Figure 6: Eigenvalues of Correlation Matrix}
\textbf{Comparison:} Shows eigenvalue spectra under three different channel models:
\begin{itemize}
    \item Jakes' model (reference)
    \item Clarke's model
    \item Block-diagonal approximation method with $\mu^2 = 0.97$
\end{itemize}

\textbf{Parameters:} Linear FAS with $W = 4$ and $N_{rx} = 100$

\textbf{Finding:} Block-diagonal approximation closely tracks Jakes' model while maintaining analytical tractability.

\subsection*{Figure 7: Outage Probability for 3-User Slow FAMA}
\textbf{Setup:} 3-user slow fluid antenna multiple access system with:
\begin{itemize}
    \item Linear FAS at each user: $W = 7, N_{rx} = 150$
    \item Comparison of three correlation models
\end{itemize}

\noindent
\textbf{Results:} Block-diagonal model provides tight approximation to Jakes' model, while constant correlation model significantly overestimates performance.

\subsection*{Figure 8: Key Considerations in FAS Models}
\textbf{Framework:} Presents essential factors influencing FAS system models:
\begin{itemize}
    \item Antenna architecture (1D, 2D, 3D)
    \item Circuit configuration (single/multiple active elements, isolation)
    \item Spatial correlation (geometric/mathematical approaches)
    \item Environmental characteristics (LoS/NLoS, scattering, frequency)
    \item Modeling accuracy vs. analytical tractability tradeoff
\end{itemize}

\subsection*{Figure 9: UAMA Architecture for CSI Extrapolation}
\textbf{Components:}
\begin{itemize}
    \item \textbf{Input:} Real and imaginary parts of observable CSI
    \item \textbf{Pre-mapper:} Non-linear projection, position encoding, MLP
    \item \textbf{Encoder:} MetaMixer blocks (Transformer, Spatial MLP, Dynamic FFT)
    \item \textbf{Mid-mapper:} Dimensionality reduction
    \item \textbf{Decoder:} MetaDiffusion blocks (CNN-like, Local Attention, GNN-like)
    \item \textbf{Post-mapper:} Output dimensionality reduction
    \item \textbf{Output:} Predicted unknown CSI
\end{itemize}

\noindent
\textbf{Architecture Elements:}
\begin{itemize}
    \item Isotropic Encoder Architecture with MetaMixer for global receptive field
    \item Isotropic Lightweight Decoder Architecture with MetaDiffusion for local correlations
    \item Multiple normalization techniques (LayerNorm, BatchNorm, InstanceNorm)
\end{itemize}

\subsection*{Figure 10: NMSE for CSI Extrapolation vs Observable Ports}
\textbf{Setup:} 
\begin{itemize}
    \item FAS dimensions: $(N_1, N_2) = (20, 40)$
    \item Physical size: $(W_1, W_2) = (2 cm, 4 cm)$
    \item Number of users: $U = 10$
    \item Frequency range: 2.5 GHz to 39 GHz
\end{itemize}

\noindent
\textbf{Key Results:}
\begin{itemize}
    \item NMSE decreases with increasing number of observable ports
    \item Achieves NMSE $\leq 10^{-3}$ with only 100 observable ports out of 800
    \item Demonstrates feasibility of CSI extrapolation across broad frequency range
\end{itemize}


\subsection*{Figure 11: Channel Estimation Setup for Multiuser Uplink}
\textbf{System Architecture:}
\begin{itemize}
    \item BS: Multiple fixed-position antennas (separated by $Δ = λ/2$)
    \item Users: Each equipped with linear FAS (N ports uniformly distributed over length Wλ)
    \item Uplink channel from user's FAS ports to BS
\end{itemize}

\subsection*{Figure 12-13: Channel Estimation Performance Metrics}
\textbf{Comparison Metrics:}
\begin{itemize}
    \item NMSE (Normalized Mean Squared Error)
    \item Computational complexity
    \item Estimation overhead
    \item Pilot sequence requirements
    \item Various estimation methods: Least squares, L3SCR, OMP, Bayesian
\end{itemize}



\section*{SECTION 2: TABLES AND TABULAR DATA}

% \subsection*{Table I: Summary of Different Review Papers on FAS}

% | Reference | Focus Areas | Key Contributions |
% |-----------|-----------|------------------|
% | [33] | Channel models, theoretical performance, hardware designs | Simplified models, six research topics, hardware implementations |
% | [38] | Gallium-based liquid metals | Physical, chemical, biological properties; application compatibility |
% | [39] | Liquid antenna design, fabrication, performance | Materials review, fabrication methods, frequency/polarization/pattern reconfiguration |
% | [40] | Metallic and non-metallic liquid antennas | State-of-art designs, real-world application challenges |
% | [41] | Liquid antenna arrays, integration challenges | Array technologies, integration challenges and solutions |
% | [67] | Flexible-position MIMO systems | Channel hardening, spectral efficiency, energy efficiency optimization |
% | [79] | Movable antenna systems | Hardware design, opportunities, challenges in estimation |
% | [83] | RIS, MIMO, and FAS synergies | Conventional RIS, surface-wave RIS, combined performance |
% | [84] | Channel models and theoretical performance | Recent channel models, MIMO-FAS and FAMA analysis |
% | [85] | Research opportunities | CAP-MIMO, MIMO-FAMA, wireless power transfer, physical layer security |
% | [86] | Distributed scattering surfaces | RIS as distributed artificial scatterers for massive connectivity |
% | This Paper | Comprehensive FAS tutorial | Communication theory, optimization, hardware, challenges, synergies |


% Please add the following required packages to your document preamble:
% \usepackage[table,xcdraw]{xcolor}
% Beamer presentation requires \usepackage{colortbl} instead of \usepackage[table,xcdraw]{xcolor}
\begin{table}[ht!]
\caption{Table I: Summary of Different Review Papers on FAS}
\label{tab:my-table}
\begin{tabular}{lll}
\rowcolor[HTML]{FFFFFF} 
\multicolumn{1}{c}{\cellcolor[HTML]{FFFFFF}{\color[HTML]{1F2328} \textbf{Reference}}} &
  \multicolumn{1}{c}{\cellcolor[HTML]{FFFFFF}{\color[HTML]{1F2328} \textbf{Focus Areas}}} &
  \multicolumn{1}{c}{\cellcolor[HTML]{FFFFFF}{\color[HTML]{1F2328} \textbf{Key Contributions}}} \\
\rowcolor[HTML]{FFFFFF} 
{\color[HTML]{1F2328} {[}33{]}} &
  {\color[HTML]{1F2328} Channel models, theoretical performance, hardware designs} &
  {\color[HTML]{1F2328} Simplified models, six research topics, hardware implementations} \\
\rowcolor[HTML]{F6F8FA} 
{\color[HTML]{1F2328} {[}38{]}} &
  {\color[HTML]{1F2328} Gallium-based liquid metals} &
  {\color[HTML]{1F2328} Physical, chemical, biological properties; application compatibility} \\
\rowcolor[HTML]{FFFFFF} 
{\color[HTML]{1F2328} {[}39{]}} &
  {\color[HTML]{1F2328} Liquid antenna design, fabrication, performance} &
  {\color[HTML]{1F2328} Materials review, fabrication methods, frequency/polarization/pattern reconfiguration} \\
\rowcolor[HTML]{F6F8FA} 
{\color[HTML]{1F2328} {[}40{]}} &
  {\color[HTML]{1F2328} Metallic and non-metallic liquid antennas} &
  {\color[HTML]{1F2328} State-of-art designs, real-world application challenges} \\
\rowcolor[HTML]{FFFFFF} 
{\color[HTML]{1F2328} {[}41{]}} &
  {\color[HTML]{1F2328} Liquid antenna arrays, integration challenges} &
  {\color[HTML]{1F2328} Array technologies, integration challenges and solutions} \\
\rowcolor[HTML]{F6F8FA} 
{\color[HTML]{1F2328} {[}67{]}} &
  {\color[HTML]{1F2328} Flexible-position MIMO systems} &
  {\color[HTML]{1F2328} Channel hardening, spectral efficiency, energy efficiency optimization} \\
\rowcolor[HTML]{FFFFFF} 
{\color[HTML]{1F2328} {[}79{]}} &
  {\color[HTML]{1F2328} Movable antenna systems} &
  {\color[HTML]{1F2328} Hardware design, opportunities, challenges in estimation} \\
\rowcolor[HTML]{F6F8FA} 
{\color[HTML]{1F2328} {[}83{]}} &
  {\color[HTML]{1F2328} RIS, MIMO, and FAS synergies} &
  {\color[HTML]{1F2328} Conventional RIS, surface-wave RIS, combined performance} \\
\rowcolor[HTML]{FFFFFF} 
{\color[HTML]{1F2328} {[}84{]}} &
  {\color[HTML]{1F2328} Channel models and theoretical performance} &
  {\color[HTML]{1F2328} Recent channel models, MIMO-FAS and FAMA analysis} \\
\rowcolor[HTML]{F6F8FA} 
{\color[HTML]{1F2328} {[}85{]}} &
  {\color[HTML]{1F2328} Research opportunities} &
  {\color[HTML]{1F2328} CAP-MIMO, MIMO-FAMA, wireless power transfer, physical layer security} \\
\rowcolor[HTML]{FFFFFF} 
{\color[HTML]{1F2328} {[}86{]}} &
  {\color[HTML]{1F2328} Distributed scattering surfaces} &
  {\color[HTML]{1F2328} RIS as distributed artificial scatterers for massive connectivity} \\
\rowcolor[HTML]{F6F8FA} 
{\color[HTML]{1F2328} This Paper} &
  {\color[HTML]{1F2328} Comprehensive FAS tutorial} &
  {\color[HTML]{1F2328} Communication theory, optimization, hardware, challenges, synergies}
\end{tabular}
\end{table}

\newpage 
\subsection*{Table II: Key Abbreviations Used in FAS Literature}

% Please add the following required packages to your document preamble:
% \usepackage[table,xcdraw]{xcolor}
% Beamer presentation requires \usepackage{colortbl} instead of \usepackage[table,xcdraw]{xcolor}
\begin{table}[ht!]
\caption{Table II: Key Abbreviations Used in FAS Literature}
\label{tab:my-table}
\begin{tabular}{ll}
\rowcolor[HTML]{FFFFFF} 
\multicolumn{1}{c}{\cellcolor[HTML]{FFFFFF}{\color[HTML]{1F2328} \textbf{Abbreviation}}} &
  \multicolumn{1}{c}{\cellcolor[HTML]{FFFFFF}{\color[HTML]{1F2328} \textbf{Definition}}} \\
\rowcolor[HTML]{FFFFFF} 
{\color[HTML]{1F2328} AoA}      & {\color[HTML]{1F2328} Angle-of-Arrival}                        \\
\rowcolor[HTML]{F6F8FA} 
{\color[HTML]{1F2328} AoD}      & {\color[HTML]{1F2328} Angle-of-Departure}                      \\
\rowcolor[HTML]{FFFFFF} 
{\color[HTML]{1F2328} AWGN}     & {\color[HTML]{1F2328} Additive White Gaussian Noise}           \\
\rowcolor[HTML]{F6F8FA} 
{\color[HTML]{1F2328} BS}       & {\color[HTML]{1F2328} Base Station}                            \\
\rowcolor[HTML]{FFFFFF} 
{\color[HTML]{1F2328} CAP-MIMO} & {\color[HTML]{1F2328} Continuous Aperture MIMO}                \\
\rowcolor[HTML]{F6F8FA} 
{\color[HTML]{1F2328} CSCG}     & {\color[HTML]{1F2328} Circularly Symmetric Complex Gaussian}   \\
\rowcolor[HTML]{FFFFFF} 
{\color[HTML]{1F2328} CSI}      & {\color[HTML]{1F2328} Channel State Information}               \\
\rowcolor[HTML]{F6F8FA} 
{\color[HTML]{1F2328} CUMA}     & {\color[HTML]{1F2328} Compact Ultra Massive Antenna Array}     \\
\rowcolor[HTML]{FFFFFF} 
{\color[HTML]{1F2328} DFT}      & {\color[HTML]{1F2328} Discrete Fourier Transform}              \\
\rowcolor[HTML]{F6F8FA} 
{\color[HTML]{1F2328} DMT}      & {\color[HTML]{1F2328} Diversity and Multiplexing Tradeoff}     \\
\rowcolor[HTML]{FFFFFF} 
{\color[HTML]{1F2328} EWOD}     & {\color[HTML]{1F2328} Electrowetting-on-Dielectric}            \\
\rowcolor[HTML]{F6F8FA} 
{\color[HTML]{1F2328} FAS}      & {\color[HTML]{1F2328} Fluid Antenna System}                    \\
\rowcolor[HTML]{FFFFFF} 
{\color[HTML]{1F2328} FAMA}     & {\color[HTML]{1F2328} Fluid Antenna Multiple Access}           \\
\rowcolor[HTML]{F6F8FA} 
{\color[HTML]{1F2328} FFT}      & {\color[HTML]{1F2328} Fast Fourier Transform}                  \\
\rowcolor[HTML]{FFFFFF} 
{\color[HTML]{1F2328} HK}       & {\color[HTML]{1F2328} Han-Kobayashi}                           \\
\rowcolor[HTML]{F6F8FA} 
{\color[HTML]{1F2328} L3SCR} &
  {\color[HTML]{1F2328} Low-Sample-Size Sparse Channel Reconstruction} \\
\rowcolor[HTML]{FFFFFF} 
{\color[HTML]{1F2328} LoS}      & {\color[HTML]{1F2328} Line-of-Sight}                           \\
\rowcolor[HTML]{F6F8FA} 
{\color[HTML]{1F2328} MIMO}     & {\color[HTML]{1F2328} Multiple-Input Multiple-Output}          \\
\rowcolor[HTML]{FFFFFF} 
{\color[HTML]{1F2328} MISO}     & {\color[HTML]{1F2328} Multiple-Input Single-Output}            \\
\rowcolor[HTML]{F6F8FA} 
{\color[HTML]{1F2328} ML}       & {\color[HTML]{1F2328} Machine Learning}                        \\
\rowcolor[HTML]{FFFFFF} 
{\color[HTML]{1F2328} MRC}      & {\color[HTML]{1F2328} Maximum Ratio Combining}                 \\
\rowcolor[HTML]{F6F8FA} 
{\color[HTML]{1F2328} NGMA}     & {\color[HTML]{1F2328} Next Generation Multiple Access}         \\
\rowcolor[HTML]{FFFFFF} 
{\color[HTML]{1F2328} NMSE}     & {\color[HTML]{1F2328} Normalized Mean Squared Error}           \\
\rowcolor[HTML]{F6F8FA} 
{\color[HTML]{1F2328} NLoS}     & {\color[HTML]{1F2328} Non Line-of-Sight}                       \\
\rowcolor[HTML]{FFFFFF} 
{\color[HTML]{1F2328} NOMA}     & {\color[HTML]{1F2328} Non-Orthogonal Multiple Access}          \\
\rowcolor[HTML]{F6F8FA} 
{\color[HTML]{1F2328} OMA}      & {\color[HTML]{1F2328} Orthogonal Multiple Access}              \\
\rowcolor[HTML]{FFFFFF} 
{\color[HTML]{1F2328} OMP}      & {\color[HTML]{1F2328} Orthogonal Matching Pursuit}             \\
\rowcolor[HTML]{F6F8FA} 
{\color[HTML]{1F2328} RF}       & {\color[HTML]{1F2328} Radio Frequency}                         \\
\rowcolor[HTML]{FFFFFF} 
{\color[HTML]{1F2328} RIS}      & {\color[HTML]{1F2328} Reconfigurable Intelligent Surfaces}     \\
\rowcolor[HTML]{F6F8FA} 
{\color[HTML]{1F2328} RSMA}     & {\color[HTML]{1F2328} Rate-Splitting Multiple Access}          \\
\rowcolor[HTML]{FFFFFF} 
{\color[HTML]{1F2328} RZF}      & {\color[HTML]{1F2328} Regularized Zero-Forcing}                \\
\rowcolor[HTML]{F6F8FA} 
{\color[HTML]{1F2328} SIC}      & {\color[HTML]{1F2328} Successive Interference Cancellation}    \\
\rowcolor[HTML]{FFFFFF} 
{\color[HTML]{1F2328} SIMO}     & {\color[HTML]{1F2328} Single-Input Multiple-Output}            \\
\rowcolor[HTML]{F6F8FA} 
{\color[HTML]{1F2328} SINR}     & {\color[HTML]{1F2328} Signal-to-Interference-plus-Noise Ratio} \\
\rowcolor[HTML]{FFFFFF} 
{\color[HTML]{1F2328} SISO}     & {\color[HTML]{1F2328} Single-Input Single-Output}              \\
\rowcolor[HTML]{F6F8FA} 
{\color[HTML]{1F2328} SNR}      & {\color[HTML]{1F2328} Signal-to-Noise Ratio}                   \\
\rowcolor[HTML]{FFFFFF} 
{\color[HTML]{1F2328} TAS}      & {\color[HTML]{1F2328} Traditional Antenna System}              \\
\rowcolor[HTML]{F6F8FA} 
{\color[HTML]{1F2328} TIN}      & {\color[HTML]{1F2328} Treating Interference as Noise}          \\
\rowcolor[HTML]{FFFFFF} 
{\color[HTML]{1F2328} UAMA}     & {\color[HTML]{1F2328} Unified Asymmetric Masked Autoencoder}   \\
\rowcolor[HTML]{F6F8FA} 
{\color[HTML]{1F2328} XL-MIMO}  & {\color[HTML]{1F2328} Extremely Large-Scale MIMO}             
\end{tabular}
\end{table}

\newpage

\subsection*{Table III: Summary of Papers on Channel Estimation in FAS}

% Please add the following required packages to your document preamble:
% \usepackage[table,xcdraw]{xcolor}
% Beamer presentation requires \usepackage{colortbl} instead of \usepackage[table,xcdraw]{xcolor}
\begin{table}[ht!]
\caption{Table III: Summary of Papers on Channel Estimation in FAS}
\label{tab:my-table}
\begin{tabular}{llll}
\rowcolor[HTML]{FFFFFF} 
\multicolumn{1}{c}{\cellcolor[HTML]{FFFFFF}{\color[HTML]{1F2328} \textbf{Reference}}} &
  \multicolumn{1}{c}{\cellcolor[HTML]{FFFFFF}{\color[HTML]{1F2328} \textbf{System Setup}}} &
  \multicolumn{1}{c}{\cellcolor[HTML]{FFFFFF}{\color[HTML]{1F2328} \textbf{Method}}} &
  \multicolumn{1}{c}{\cellcolor[HTML]{FFFFFF}{\color[HTML]{1F2328} \textbf{Key Contribution}}} \\
\rowcolor[HTML]{FFFFFF} 
{\color[HTML]{1F2328} {[}126{]}} &
  {\color[HTML]{1F2328} Multi-cell network with ring FAS at UEs} &
  {\color[HTML]{1F2328} LMMSE-based method} &
  {\color[HTML]{1F2328} Skipped-enabled channel estimation (SeCE) technique} \\
\rowcolor[HTML]{F6F8FA} 
{\color[HTML]{1F2328} {[}150{]}} &
  {\color[HTML]{1F2328} Multiuser uplink mmWave} &
  {\color[HTML]{1F2328} Low-sample-size sparse channel reconstruction (L3SCR)} &
  {\color[HTML]{1F2328} Reduced overhead with sparse parameter estimation} \\
\rowcolor[HTML]{FFFFFF} 
{\color[HTML]{1F2328} {[}151{]}} &
  {\color[HTML]{1F2328} Point-to-point mmWave} &
  {\color[HTML]{1F2328} Least squares regression} &
  {\color[HTML]{1F2328} Parameter estimation in mmWave bands} \\
\rowcolor[HTML]{F6F8FA} 
{\color[HTML]{1F2328} {[}152{]}} &
  {\color[HTML]{1F2328} Point-to-point with 2D FAS} &
  {\color[HTML]{1F2328} Successive transmitter-receiver compressed sensing (STRCS)} &
  {\color[HTML]{1F2328} 2D FAS channel reconstruction} \\
\rowcolor[HTML]{FFFFFF} 
{\color[HTML]{1F2328} {[}153{]}} &
  {\color[HTML]{1F2328} Point-to-point with 2D FAS} &
  {\color[HTML]{1F2328} Orthogonal matching pursuit (OMP)} &
  {\color[HTML]{1F2328} Reduced error propagation, lower overhead} \\
\rowcolor[HTML]{F6F8FA} 
{\color[HTML]{1F2328} {[}154{]}} &
  {\color[HTML]{1F2328} Point-to-point with linear FAS} &
  {\color[HTML]{1F2328} Successive Bayesian reconstructor (S-BAR)} &
  {\color[HTML]{1F2328} Bayesian-based sparse parameter estimation}
\end{tabular}
\end{table}



\subsection*{FAS Performance Comparison with Traditional Systems}
% Please add the following required packages to your document preamble:
% \usepackage[table,xcdraw]{xcolor}
% Beamer presentation requires \usepackage{colortbl} instead of \usepackage[table,xcdraw]{xcolor}
\begin{table}[ht!]
\caption{FAS Performance Comparison with Traditional Systems}
\label{tab:my-table}
\begin{tabular}{lll}
\rowcolor[HTML]{FFFFFF} 
\multicolumn{1}{c}{\cellcolor[HTML]{FFFFFF}{\color[HTML]{1F2328} \textbf{Metric}}} &
  \multicolumn{1}{c}{\cellcolor[HTML]{FFFFFF}{\color[HTML]{1F2328} \textbf{FAS}}} &
  \multicolumn{1}{c}{\cellcolor[HTML]{FFFFFF}{\color[HTML]{1F2328} \textbf{Traditional Antenna Systems (TAS)}}} \\
\rowcolor[HTML]{FFFFFF} 
{\color[HTML]{1F2328} \textbf{Physical Antennas Required}} &
  {\color[HTML]{1F2328} Single or few} &
  {\color[HTML]{1F2328} Multiple (one per port/position)} \\
\rowcolor[HTML]{F6F8FA} 
{\color[HTML]{1F2328} \textbf{RF-Chains Required}} &
  {\color[HTML]{1F2328} Fewer} &
  {\color[HTML]{1F2328} More} \\
\rowcolor[HTML]{FFFFFF} 
{\color[HTML]{1F2328} \textbf{Hardware Complexity}} &
  {\color[HTML]{1F2328} Lower} &
  {\color[HTML]{1F2328} Higher} \\
\rowcolor[HTML]{F6F8FA} 
{\color[HTML]{1F2328} \textbf{Power Consumption}} &
  {\color[HTML]{1F2328} Lower} &
  {\color[HTML]{1F2328} Higher} \\
\rowcolor[HTML]{FFFFFF} 
{\color[HTML]{1F2328} \textbf{Spatial Diversity}} &
  {\color[HTML]{1F2328} Exploits entire designated space} &
  {\color[HTML]{1F2328} Limited by discrete antenna spacing $(\geq \lambda/2)$} \\
\rowcolor[HTML]{F6F8FA} 
{\color[HTML]{1F2328} \textbf{Channel Hardening}} &
  {\color[HTML]{1F2328} Enhanced with fewer radiating elements} &
  {\color[HTML]{1F2328} Requires many antennas} \\
\rowcolor[HTML]{FFFFFF} 
{\color[HTML]{1F2328} \textbf{Spectral Efficiency}} &
  {\color[HTML]{1F2328} Higher} &
  {\color[HTML]{1F2328} Lower for same number of RF-chains} \\
\rowcolor[HTML]{F6F8FA} 
{\color[HTML]{1F2328} \textbf{Energy Efficiency}} &
  {\color[HTML]{1F2328} Higher} &
  {\color[HTML]{1F2328} Lower} \\
\rowcolor[HTML]{FFFFFF} 
{\color[HTML]{1F2328} \textbf{Position Reconfiguration}} &
  {\color[HTML]{1F2328} Yes (software-controlled)} &
  {\color[HTML]{1F2328} No (fixed positions)} \\
\rowcolor[HTML]{F6F8FA} 
{\color[HTML]{1F2328} \textbf{Frequency Adaptability}} &
  {\color[HTML]{1F2328} Yes (shape/size reconfiguration)} &
  {\color[HTML]{1F2328} No} \\
\rowcolor[HTML]{FFFFFF} 
{\color[HTML]{1F2328} \textbf{Polarization Control}} &
  {\color[HTML]{1F2328} Yes (orientation control)} &
  {\color[HTML]{1F2328} Limited}
\end{tabular}
\end{table}

\newpage 

\section*{SECTION 3: EQUATIONS AND MATHEMATICAL MODELS}

\subsection*{A. Simplified Channel Model}

\subsubsection*{Equation 1: Correlated Channel Coefficients (Reference Port Method)}
$$h_n = \sqrt{1 - \mu_n^2 x_n + \mu_n x_1} + j\sqrt{1 - \mu_n^2 y_n + \mu_n y_1}, \quad n = 2, \ldots, N$$

\textbf{Parameters:}
\begin{itemize}
    \item $\mu_n$: Correlation parameter (defined below)
    \item $x_1, \ldots, x_N, y_1, \ldots, y_N$: i.i.d. real Gaussian variables with zero-mean and variance $1/2$
    \item Reference port: 1st port
\end{itemize}

\subsubsection*{Equation 2: Correlation Parameter Using Bessel Function}
$$\mu_n = J_0\left(2\pi \frac{|n-1|}{N-1} W\right)$$

\textbf{Where:}
\begin{itemize}
    \item $J_0(\cdot)$: Zero-order Bessel function of the first kind
    \item $W$: Normalized antenna aperture length (in wavelengths)
    \item $N$: Total number of ports
\end{itemize}

\subsubsection*{Equation 3: Common Correlation Parameter (Reference Port-Free)}
$$\mu = \sqrt{2} \left( {}_1F_2\left(\frac{1}{2}; 1, \frac{3}{2}; -\pi^2 W^2\right) - \frac{J_1(2\pi W)}{2\pi W}\right)$$

\textbf{Where:}
\begin{itemize}
    \item ${}_1F_2(\cdot; \cdot; \cdot)$: Generalized hypergeometric function
    \item $J_1(\cdot)$: First-order Bessel function of the first kind
    \item Advantage: Links all ports without a reference port
\end{itemize}

\subsubsection*{Equation 4: Traditional Antenna System (TAS) Channel}
$$h_1 = x_1 + jy_1$$

\textbf{Comparison:} Equivalent to having one port in FAS, showing how TAS limits channel dimensions


\newpage 
\subsection*{B. Fully Correlated Channel Model (Jakes' Model)}

\subsubsection*{Equation 5: Covariance Matrix Element}
$$J_{n,m} = \text{Cov}[h_n, h_m] = J_0\left(2\pi \frac{|n-m|}{N-1} W\right)$$

\textbf{Characteristic:} Full correlation structure based on port spacing

\subsubsection*{Equation 6: Channel Vector via Eigenvalue Decomposition}
$$\mathbf{h} = \mathbf{Q}\Lambda^{1/2} \mathbf{g}$$

\textbf{Where:}
\begin{itemize}
    \item $\mathbf{J} = \mathbf{Q}\Lambda\mathbf{Q}^H$: Eigenvalue decomposition of covariance matrix
    \item $\mathbf{g} = [g_1, \ldots, g_N]^T$: CSCG random variables with $g_n \sim \mathcal{CN}(0,1)$
\end{itemize}

\subsubsection*{Equation 7: Channel Coefficient as Linear Combination}
$$h_n = \sum_{m=1}^{N} q_{n,m}\sqrt{\lambda_m} (x_m + jy_m)$$

\textbf{Properties:}
\begin{itemize}
    \item $q_{n,m}$: Elements of eigenvector matrix $\mathbf{Q}$
    \item $\lambda_m$: Eigenvalues (ordered: $\lambda_1 \geq \cdots \geq \lambda_N$)
    \item Follows Jakes' assumption: $\mathbf{h} \sim \mathcal{CN}(0, \mathbf{J})$
\end{itemize}

\subsubsection*{Equation 8: Low-Rank Approximation of Channel Coefficient}
$$\hat{h}_n = \sum_{m=1}^{\hat{N}} q_{n,m}\sqrt{\lambda_m} (x_m + jy_m)$$

\textbf{Where:} $\hat{N} \ll N$ (uses only dominant eigenvalues)

\subsubsection*{Equation 9: Average Variance of Approximated Coefficient}
$$\frac{1}{N}\sum_{n=1}^{N} \text{Cov}\left(\hat{h}_n\right) = \frac{1}{N}\sum_{n=1}^{N}\sum_{m=1}^{\hat{N}} q_{n,m}^2 \lambda_m$$


\newpage 
\subsection*{C. Channel Model for 2D FAS}

\subsubsection*{Equation 10: Mapping 2D Indices to 1D Index}
$$k_{(n_1, n_2)} = (n_2 - 1) N_1 + n_1$$

\textbf{Application:} Converts 2D array coordinates to single linear index for analysis

\subsubsection*{Equation 11: Covariance Matrix for 2D FAS (Transmitter)}
$$\mathbf{J}^{tx} = \begin{bmatrix}
J^{tx}_{1,1} & \cdots & J^{tx}_{1,k_{(\tilde{n}_1,\tilde{n}_2)}} & \cdots & J^{tx}_{1,N^{tx}} \\
\vdots & \ddots & \vdots & \ddots & \vdots \\
J^{tx}_{k_{(n_1,n_2)},1} & \cdots & J^{tx}_{k_{(n_1,n_2)},k_{(\tilde{n}_1,\tilde{n}_2)}} & \cdots & J^{tx}_{k_{(n_1,n_2)},N^{tx}} \\
\vdots & \ddots & \vdots & \ddots & \vdots \\
J^{tx}_{N^{tx},1} & \cdots & J^{tx}_{N^{tx},k_{(\tilde{n}_1,\tilde{n}_2)}} & \cdots & J^{tx}_{N^{tx},N^{tx}}
\end{bmatrix}$$

\subsubsection*{Equation 12: Spatial Correlation for 2D FAS in 3D Environment}
$$J^{tx}_{k_{(n_1,n_2)},k_{(\tilde{n}_1,\tilde{n}_2)}} = j_0\left(2\pi\sqrt{\left(\frac{|n_1-\tilde{n}_1|}{N^{tx}_1-1} W^{tx}_1\right)^2 + \left(\frac{|n_2-\tilde{n}_2|}{N^{tx}_2-1} W^{tx}_2\right)^2}\right)$$

\textbf{Where:}
\begin{itemize}
    \item $j_0(\cdot)$: Zero-order spherical Bessel function or sinc function
    \item $W^s_i$: Normalized aperture dimensions
    \item Rich scattering environment assumption
\end{itemize}

\subsubsection*{Equation 13: Channel Matrix for 2D FAS at Both Ends}
$$\mathbf{H} = \mathbf{Q}_{rx}\Lambda_{rx}^{1/2} \mathbf{G} (\Lambda_{tx}^{1/2})^H \mathbf{Q}_{tx}^H$$

\textbf{Components:}
\begin{itemize}
    \item $\mathbf{G} \in \mathbb{C}^{N_{rx} \times N_{tx}}$: CSCG random matrix with i.i.d. entries $\sim \mathcal{CN}(0,1)$
    \item Both transmitter and receiver have 2D FAS surfaces
\end{itemize}


\newpage 
\subsection*{D. Block Spatial Correlation Model}

\subsubsection*{Equation 14: Block-Diagonal Approximation of Covariance Matrix}
$$\hat{\mathbf{J}} = \begin{bmatrix}
A_1 & 0 & \cdots & 0 \\
0 & A_2 & \cdots & 0 \\
\vdots & \vdots & \ddots & \vdots \\
0 & 0 & \cdots & A_B
\end{bmatrix}$$

\textbf{Where:}
\begin{itemize}
    \item Each block $A_b$: Constant correlation matrix of size $L_b$
    \item Correlation parameter: $\mu_b^2$ (often $\mu_b = \mu$ for all blocks)
\end{itemize}

\subsubsection*{Equation 15: Eigenvalues of Block Correlation Matrix}
$$\hat{\lambda}_{n'} = \begin{cases}
(L_b - 1)\mu^2 + 1, & \text{if } n' = 1 \\
1 - \mu^2, & \text{if } n' = 2, \ldots, L_b
\end{cases}$$

\subsubsection*{Equation 16: Block Size Determination}
$$L_b = \left\lfloor \frac{\lambda_b - 1}{\mu^2} + 1 \right\rfloor$$

\textbf{Where:} $\{\lambda_b\}_{b=1}^B$ are dominant eigenvalues of true correlation matrix


\newpage 
\subsection*{E. Finite Scattering Channel Model}

\subsubsection*{Equation 17: Planar-Wave Geometric Channel Model}
$$\mathbf{H} = \sqrt{\frac{K}{K+1}} e^{j\omega} \mathbf{a}_r(\theta_{0,r}, \phi_{0,r}) \mathbf{a}_t(\theta_{0,t}, \phi_{0,t})^H + \sqrt{\frac{1}{L_p(K+1)}} \sum_{l=1}^{L_p} \kappa_l \mathbf{a}_r(\theta_{l,r}, \phi_{l,r}) \mathbf{a}_t(\theta_{l,t}, \phi_{l,t})^H$$

\textbf{Components:}
\begin{itemize}
    \item $K$: Rice factor (LoS strength)
    \item $\omega$: Phase of LoS component
    \item $\kappa_l$: Complex channel gain of l-th scattered component
    \item $L_p$: Total number of NLoS paths
    \item $\mathbf{a}_r, \mathbf{a}_t$: Receive and transmit steering vectors
\end{itemize}

\subsubsection*{Equation 18: Receive Steering Vector}
$$\mathbf{a}_r(\theta_r, \phi_r) = \left[1, e^{-j\frac{2\pi}{\lambda}\psi(\theta_r,\phi_r)^T \mathbf{n}^{rx}_{1\lambda}}, \ldots, e^{-j\frac{2\pi}{\lambda}\psi(\theta_r,\phi_r)^T \mathbf{n}^{rx}_{N_{rx}\lambda}}\right]^T$$

\subsubsection*{Equation 19: 3D Receiver Port Positions}
$$\mathbf{n}^{rx}_n = \left(\frac{n^{rx}_3-1}{N^{rx}_3-1}W^{rx}_3, \frac{n^{rx}_2-1}{N^{rx}_2-1}W^{rx}_2, \frac{n^{rx}_1-1}{N^{rx}_1-1}W^{rx}_1\right)$$

\subsubsection*{Equation 20: Wave Vector Definition}
$$\boldsymbol{\psi}(\theta, \phi) = \begin{bmatrix}
\cos\phi\cos\theta \\
\cos\phi\sin\theta \\
\sin\phi
\end{bmatrix}$$

\subsubsection*{Equations 21-22: 2D and 1D FAS Port Positions}
\textbf{2D case:}
$$\mathbf{n}^{rx}_n = \left(0, \frac{n^{rx}_2-1}{N^{rx}_2-1}W^{rx}_2, \frac{n^{rx}_1-1}{N^{rx}_1-1}W^{rx}_1\right)$$

\textbf{1D case:}
$$\mathbf{n}^{rx}_n = \left(0, 0, \frac{n^{rx}_1-1}{N^{rx}_1-1}W^{rx}_1\right)$$


\newpage 
\subsection*{F. Copula-Based Channel Model}

\subsubsection*{Equation 23: N-Dimensional Copula Definition}
$$C(b_1, \ldots, b_N; \vartheta_C) = \Pr(B_1 \leq b_1, \ldots, B_N \leq b_N)$$

\textbf{Where:}
\begin{itemize}
    \item $b_n = F_{|h_n|}(r_n)$: Mapped to uniform distribution
    \item $\vartheta_C$: Copula parameter measuring dependency
\end{itemize}

\subsubsection*{Equation 24: Sklar's Theorem (Copula Representation)}
$$F_{|h_1|,\ldots,|h_N|}(r_1, \ldots, r_N) = C\left(F_{|h_1|}(r_1), \ldots, F_{|h_N|}(r_N)\right)$$

\subsubsection*{Equation 25: Joint PDF via Copula}
$$f_{|h_1|,\ldots,|h_N|}(r_1, \ldots, r_N) = \frac{\partial^N C(F_{|h_1|}(r_1), \ldots, F_{|h_N|}(r_N); \vartheta_C)}{\partial F_{|h_1|}(r_1) \cdots \partial F_{|h_N|}(r_N)} \prod_{n=1}^{N} f_{|h_n|}(r_n)$$

\subsection*{G. Circuit and Antenna Models}

\subsubsection*{Equation 26: Equivalent Channel with Multiple Active Ports}
$$\bar{\mathbf{H}} = \mathbf{A}_{rx} \mathbf{H} \mathbf{A}_{tx}$$

\textbf{Where:}
\begin{itemize}
    \item $\mathbf{A}_{tx} = [\boldsymbol{\alpha}^{tx}_1, \ldots, \boldsymbol{\alpha}^{tx}_{n^{tx}}]$: Activation port matrix at transmitter
    \item $\mathbf{A}_{rx} = [\boldsymbol{\alpha}^{rx}_1, \ldots, \boldsymbol{\alpha}^{rx}_{n^{rx}}]^T$: Activation port matrix at receiver
    \item $\boldsymbol{\alpha}^s_l \in \{\mathbf{e}_1, \ldots, \mathbf{e}_{N^s}\}$: Standard basis vectors
\end{itemize}

\subsubsection*{Equation 27: Channel with Mutual Coupling Effect}
$$\bar{\mathbf{H}}_{mc} = \mathbf{Z}^{rx}_{mc} \bar{\mathbf{H}} \mathbf{Z}^{tx}_{mc}$$

\textbf{Where:}
\begin{itemize}
    \item $\mathbf{Z}^{rx}_{mc}, \mathbf{Z}^{tx}_{mc}$: Mutual coupling matrices
\end{itemize}

\subsubsection*{Equation 28: Mutual Coupling Matrix Computation}
$$\mathbf{Z}^s_{mc} = (\mathbf{Z}^s_A + \mathbf{Z}^s_L) (\mathbf{Z}^s + \mathbf{Z}^s_{LI})^{-1}$$

\textbf{Components:}
\begin{itemize}
    \item $\mathbf{Z}^s_A$: Antenna impedance
    \item $\mathbf{Z}^s_L$: Load impedance
    \item $\mathbf{Z}^s$: Mutual impedance matrix
\end{itemize}

\subsubsection*{Equation 29: Relationship Between Impedance and Scattering Parameters}
$$\mathbf{Z}^s_{mc} = Z_0 (\mathbf{I} - \mathbf{S}^s)^{-1} (\mathbf{I} + \mathbf{S}^s)$$

\textbf{Where:}
\begin{itemize}
    \item $Z_0$: Reference impedance
    \item $\mathbf{S}^s$: Scattering parameter matrix
\end{itemize}

\subsubsection*{Equation 30: Spatial Correlation for Traditional Antenna Systems}
$$J^s_{k(m_1,m_2),k(\tilde{m}_1,\tilde{m}_2)} = j_0\left(2\pi\sqrt{(|m_1-\tilde{m}_1|d^s_1)^2 + (|m_2-\tilde{m}_2|d^s_2)^2}\right)$$

\textbf{Where:}
\begin{itemize}
    \item $d^s_i$: Antenna separation distances (constraint: $d^s_i \geq 0.5\lambda$)
    \item $m_i \in \{1, \ldots, \lfloor W^s_i / d^s_i \rfloor + 1\}$
\end{itemize}


\newpage 
\section*{SECTION 4: CHANNEL MODELS}

\subsection*{Overview of Channel Models for FAS}

The paper discusses multiple channel models with varying levels of complexity and accuracy:

\subsubsection*{1. Simplified Channel Model}
\begin{itemize}
    \item \textbf{Advantage:} Analytically tractable
    \item \textbf{Disadvantage:} May overestimate performance
    \item \textbf{Use Case:} Theoretical analysis and simplified studies
\end{itemize}

\subsubsection*{2. Fully Correlated (Jakes') Channel Model}
\begin{itemize}
    \item \textbf{Advantage:} Accurate, realistic propagation assumptions
    \item \textbf{Disadvantage:} Computationally intractable without approximations
    \item \textbf{Use Case:} Simulation studies, performance validation
\end{itemize}

\subsubsection*{3. Block Spatial Correlation Model (Novel contribution)}
\begin{itemize}
    \item \textbf{Advantage:} Balances accuracy and tractability
    \item \textbf{Disadvantage:} Requires eigenvalue analysis
    \item \textbf{Use Case:} Both analytical and simulation studies
    \item \textbf{Key Benefit:} Tight approximation to Jakes' model with minimal complexity
\end{itemize}

\subsubsection*{4. Finite Scattering Channel Model}
\begin{itemize}
    \item \textbf{Advantage:} Accounts for LoS, NLoS, 3D environments, specific scattering paths
    \item \textbf{Disadvantage:} Difficult analytical characterization
    \item \textbf{Use Case:} Simulation, practical scenarios
\end{itemize}

\subsubsection*{5. Copula-Based Channel Model}
\begin{itemize}
    \item \textbf{Advantage:} Captures non-linear correlations, works with arbitrary fading distributions
    \item \textbf{Disadvantage:} Determining appropriate copula function can be challenging
    \item \textbf{Use Case:} Non-ideal conditions, non-linear dependencies
\end{itemize}

\subsection*{Key Factors Influencing System Models}

\begin{enumerate}
    \item \textbf{Antenna Architecture:} 1D, 2D, 3D fluid antenna surfaces
    \item \textbf{Circuit Configuration:} Single vs. multiple active elements, isolation techniques
    \item \textbf{Spatial Correlation:} Position-dependent correlation structure
    \item \textbf{Environment Type:} Rich scattering vs. finite scattering, LoS vs. NLoS
    \item \textbf{Frequency Range:} Determines near-field vs. far-field effects
    \item \textbf{Modeling Trade-off:} Accuracy vs. analytical tractability
\end{enumerate}


\noindent
\section*{SECTION 5: CHANNEL ESTIMATION METHODS}
\noindent
\subsection*{A. Rich Scattering Environment - Deep Learning Approach}

\noindent
\subsubsection*{Unified Asymmetric Masked Autoencoder (UAMA) Architecture}

\textbf{Key Benefit:} Exploit strong spatial correlation to estimate full CSI from partial observations

\textbf{Mathematical Formulation:}

$$h_U = f_{net}(h_O, \Theta)$$

Where:
\begin{itemize}
    \item $h_O$: Observable port CSI
    \item $h_U$: Unknown port CSI
    \item $\Theta = \{\theta_P, \theta_E, \theta_M, \theta_D, \theta_R\}$: Learnable parameters
\end{itemize}

\vspace{0.5cm}
\noindent
\textbf{UAMA Composition:}
$$f^{(\Theta)}_{net} = \text{Post-mapper}(\theta_R) \circ \text{Decoder}(\theta_D) \circ \text{Mid-mapper}(\theta_M) \circ \text{Encoder}(\theta_E) \circ \text{Pre-mapper}(\theta_P)$$

\newpage 
\textbf{Key Modules:}

\begin{enumerate}
    \item \textbf{Pre-mapper ($\theta_P$):} Non-linear projection, position encoding
     \begin{itemize}
    \item Multi-layer perceptron (MLP)
    \item Non-linear activation (GELU, RELU)
    \item Positional encoding techniques: absolute, relative, or learnable
     \end{itemize}
     
    \item \textbf{Encoder ($\theta_E$):} Basis vector construction
     \begin{itemize}
    \item MetaMixer mechanisms with global receptive field
    \item Attention mechanisms
    \item Spatial MLP (DynaMixer)
    \item Dynamic FFT (FNet)
     \end{itemize}
     
    \item \textbf{Mid-mapper ($\theta_M$):} Dimensionality reduction of basis vectors
     
    \item \textbf{Decoder ($\theta_D$):} Unknown CSI recovery
     \begin{itemize}
    \item MetaDiffusion architectures with local correlations
    \item CNN-like (ResNet)
    \item Local Attention (Swin)
    \item Graph Neural Networks (GPS)
     \end{itemize}
   
    \item \textbf{Post-mapper ($\theta_R$):} Output dimensionality reduction
\end{enumerate}

\textbf{Complexity Analysis:}
\begin{itemize}
    \item Attention mechanism: $O(N_O^2 l_1 + N_O l_1^2)$
    \item CNN-like mechanism: $O(N c_o^2 l_2^2 + N l_2^2)$
    \item Overall: $O(N_O^2 l_1 + N_O l_1^2 + N c_o^2 l_2^2 + N l_2^2)$ where $N_O, c_o \ll N$
\end{itemize}

\vspace{0.5cm}
\textbf{Performance Metric:}
$$\text{NMSE}_{h_U,\hat{h}_U} = \frac{\sum_{t=1}^{S_{test}} \|h_U^{(t)} - \hat{h}_U^{(t)}\|^2}{\sum_{t=1}^{S_{test}} \|h_U^{(t)}\|^2}$$

\textbf{Objective Function:} Mean Squared Error (MSE)

\newpage 
\subsection*{B. Finite Scattering Environment - Mathematical Approaches}

\subsubsection*{1. Least Squares Method}

\textbf{Setup:} Multiuser uplink with sparse channel structure

\textbf{Channel Matrix:}
$$\mathbf{H}_u = [\mathbf{h}_{u,1}, \ldots, \mathbf{h}_{u,N}]$$

\textbf{Least Squares Estimate:}
$$\hat{\mathbf{H}}_{u,LS} = \mathbf{H}_u + \hat{\Sigma}_u$$

\textbf{Drawback:} Requires switching through all N ports, high pilot overhead

\subsubsection*{2. Low-Sample-Size Sparse Channel Reconstruction (L3SCR) Method}

\textbf{Principle:} Exploit channel sparsity in finite scattering environments

\textbf{Channel Representation:}
$$\mathbf{H}_{o,u} = \sqrt{\frac{MN_O}{L_u}} \sum_{l=1}^{L_u} \kappa_l^u \mathbf{a}_{u,r}(\theta_l^u,r) \mathbf{a}_{u,t}(\theta_l^u,t)^H$$

\textbf{Estimation Steps:}

a) \textbf{Number of Paths and AoA Estimation:}
   \begin{itemize}
       \item DFT-based method with angular rotation
       \item $$\hat{\mathbf{H}}_{u,LS}^{DFT} = \Omega^H \mathbf{A}_{u,r} \mathbf{K}_u \mathbf{A}_{u,t}^H + \frac{1}{\sqrt{MN_O}} \Omega^H \hat{\Sigma}_{o,u}$$
       \item Angular rotation matrix: $\Psi \in \mathbb{C}^{M \times M}$ with rotation parameter $\psi \in [-1/(2M), 1/(2M)]$
   \end{itemize}

b) \textbf{AoD and Channel Gains Estimation:}
   \begin{itemize}
       \item 1-sparse reconstruction via matched filters
       \item Dictionary matrix approach
   \end{itemize}

c) \textbf{Channel Reconstruction:}
   $$\hat{\mathbf{H}}_{u,L3SCR} = \sqrt{\frac{MN}{\hat{L}_u}} \sum_{l=1}^{\hat{L}_u} \hat{\kappa}_l^u \mathbf{a}_{u,r}(\hat{\theta}_l^u,r) \hat{\mathbf{a}}_{u,t}(\hat{\theta}_l^u,t)^H$$

\textbf{Advantages:} Reduced overhead ($N_O \ll N$), low complexity

\subsubsection*{3. Orthogonal Matching Pursuit (OMP) Method}

\textbf{Principle:} Iteratively select AoA-AoD pairs through correlation matching

\textbf{Process:}
\begin{enumerate}
    \item Select quantized angle grids
    \item Compute sensing matrix
    \item Iteratively choose most strongly correlated column
    \item Update residual vector
    \item Repeat until convergence threshold reached
    \item Reconstruct channel (similar to L3SCR final step)
\end{enumerate}

\textbf{Advantage:} More accurate AoA-AoD pair estimation than DFT-based method
\textbf{Disadvantage:} Higher computational complexity

\subsubsection*{4. Other Estimation Methods}

\begin{itemize}
    \item \textbf{LMMSE-based approach:} Linear minimum mean-squared error estimation
    \item \textbf{Successive Transmitter-Receiver Compressed Sensing (STRCS):} For 2D FAS
    \item \textbf{Successive Bayesian Reconstructor (S-BAR):} Bayesian-based sparse parameter estimation
\end{itemize}



\section*{SECTION 6: HARDWARE DESIGNS}

\subsection*{A. Mechanical Movable Antenna Design}

\textbf{Principle:} Physical repositioning of antenna elements using motors or actuators

\textbf{Components:}
\begin{itemize}
    \item Stepper motors or mechanical actuators
    \item Antenna elements (dipoles, patches, etc.)
    \item Control electronics for position feedback
\end{itemize}

\textbf{Advantages:}
\begin{itemize}
    \item Well-established technology
    \item Precise position control possible
    \item Low mutual coupling between inactive elements
\end{itemize}

\textbf{Disadvantages:}
\begin{itemize}
    \item Speed limitations (mechanical latency)
    \item Wear and fatigue concerns
    \item Size constraints
    \item Power consumption for actuation
\end{itemize}

\textbf{Applications:}
\begin{itemize}
    \item Wireless sensor networks
    \item IoT devices
    \item Portable communications
\end{itemize}

\subsection*{B. Liquid-Based Antenna Design}

\textbf{Principle:} Use conductive or dielectric liquid materials to dynamically reconfigure antenna shape and position

\subsubsection*{Types of Liquid Materials:}

\begin{itemize}
    \item \textbf{Gallium-Based Liquid Metals:}
\end{itemize}
\begin{itemize}
    \item Gallium (Ga)
    \item Galinstan (Ga-In-Sn alloy)
    \item Properties: High conductivity, low melting point ($\approx29\circ C $for Ga)
\end{itemize}

\textbf{Performance Range:} Operating temperature typically $20-60\circ C$

\textbf{Fabrication Methods:}
\begin{enumerate}
    \item \textbf{Direct Patterning:} Cutting liquid into desired shapes
    \item \textbf{Microfluidic Systems:} Channels for liquid flow control
    \item \textbf{EWOD (Electrowetting-on-Dielectric):} Electric field control of liquid position
\end{enumerate}

\subsubsection*{Key Parameters:}

\textbf{Gallium-based liquid metals:}
\begin{itemize}
    \item Conductivity: $\sigma \sim 3.4 \times 10^6$ S/m (slightly lower than copper)
    \item Surface tension: $\sim 0.65$ N/m at room temperature
    \item Specific gravity: $\sim 6.09$ (heavier than water)
\end{itemize}

\textbf{Reconfiguration Capabilities:}
\begin{itemize}
    \item Frequency reconfiguration: Change antenna dimensions → change resonant frequency
    \item Pattern reconfiguration: Reshape elements → change radiation pattern
    \item Polarization reconfiguration: Rotate elements → change polarization
\end{itemize}

\textbf{Advantages:}
\begin{itemize}
    \item Smooth reconfiguration without discrete switching
    \item Lower profile compared to mechanical systems
    \item Integration with wearable devices
    \item Wireless power transfer applications
\end{itemize}

\textbf{Disadvantages:}
\begin{itemize}
    \item Toxicity and environmental concerns (gallium-based)
    \item Temperature sensitivity
    \item Encapsulation challenges
    \item Migration and leakage risks
    \item Slower reconfiguration compared to electrical switching
\end{itemize}

\textbf{Integration Challenges:}
\begin{itemize}
    \item Substrate compatibility
    \item Sealing and isolation
    \item Thermal management
    \item Reliability in extreme conditions
\end{itemize}

\subsection*{C. Pixel-Based Antenna Design}

\textbf{Principle:} Dense array of individually controlled antenna elements (pixels/ports)

\subsubsection*{Architecture:}

\textbf{Reconfigurable Pixel Array (RPA):}
\begin{itemize}
    \item Grid of reconfigurable pixels
    \item Adjacent pixels can be connected or isolated
    \item Each connection state can be controlled via switches
\end{itemize}

\textbf{Configuration State Definition:}
$$\mathbf{x} = [x_1, x_2, \ldots, x_{Q_p}]$$

\noindent
Where $x_q \in \{0,1\}$ indicates whether q-th connection is open (0) or closed (1)

\textbf{Antenna Parameters Relation:}
\begin{itemize}
    \item Radiation pattern: $E(\Omega, \mathbf{x})$
    \item Input impedance: $z(\mathbf{x})$
\end{itemize}

Number of possible states: $2^{Q_p}$

\subsubsection*{FAS Port Selection:}

\textbf{Return Loss Matching Constraint:}
$$S_{feed}(\mathbf{x}_n) < -10 \text{ dB}, \quad n = 1, 2, \ldots, N_s$$

\textbf{Return Loss Definition:}
$$S_{feed}(\mathbf{x}_n) = 20\log\left|\frac{z(\mathbf{x}_n) - Z_0}{z(\mathbf{x}_n) + Z_0}\right| \text{ dB}$$

Where $Z_0 \approx 377\, \Omega$ (free space impedance)

\textbf{Optimization Problem:}
$$\min_{V_l^*} \sum_n \sum_{n'} |||\mu(\mathbf{x}_n, \mathbf{x}_{n'})| - |\mu^*(\mathbf{x}_n, \mathbf{x}_{n'})||$$

\textbf{Correlation Between States:}
$$\mu(\mathbf{x}_n, \mathbf{x}_{n'}) = \int_0^{2\pi} \int_0^{\pi} E(\Omega, \mathbf{x}_n) \cdot E(\Omega, \mathbf{x}_{n'}) d\Omega$$

\textbf{Optimization Methods:}
\begin{itemize}
    \item Heuristic techniques (genetic algorithms, simulated annealing)
    \item Exhaustive search for small codebooks
\end{itemize}

\textbf{Advantages:}
\begin{itemize}
    \item Multiple active elements possible
    \item High spatial resolution
    \item Fast electronic switching
    \item Large number of configurations
\end{itemize}

\textbf{Disadvantages:}
\begin{itemize}
    \item Mutual coupling effects (even between inactive pixels)
    \item Complex circuit design
    \item Higher component count
    \item Impedance matching complexity
    \item Non-linear effects
\end{itemize}

\textbf{Mutual Coupling Mitigation:}
\begin{itemize}
    \item Isolation techniques between pixels
    \item Circuit-level impedance compensation
    \item Adaptive matching networks
\end{itemize}

\subsection*{D. Hybrid Antenna Design}

\textbf{Principle:} Combination of multiple antenna technologies for enhanced flexibility

\textbf{Examples:}

\begin{enumerate}
    \item \textbf{Liquid + Pixel Hybrid:}
     \begin{itemize}
     \item Use liquid materials to adjust pixel conductivity
     \item Highly conductive solution for active pixels
     \item Low-conductive solution for inactive pixels
     \item Address mutual coupling dynamically
     \end{itemize}
     
    \item \textbf{Mechanical + Liquid:}
     \begin{itemize}
     \item Mechanical positioning for coarse adjustment
     \item Liquid reconfiguration for fine-tuning
     \item Combine speed (electrical) with precision (mechanical)
     \end{itemize}
     
    \item \textbf{Multi-Technology Approach:}
     \begin{itemize}
     \item Different parts using different FAS technologies
     \item Optimize each section for specific requirements
     \end{itemize}
\end{enumerate}

\textbf{Advantages:}
\begin{itemize}
    \item Superior flexibility and reconfiguration
    \item Improved performance through tailored designs
    \item Address specific application requirements
\end{itemize}

\textbf{Disadvantages:}
\begin{itemize}
    \item Higher design complexity
    \item Integration challenges
    \item Increased manufacturing difficulty
    \item Potential for unexpected interactions
    \item Cost implications
\end{itemize}


\newpage 
\section*{SECTION 7: TECHNICAL SPECIFICATIONS}

\subsection*{A. FAS Characteristics and Parameters}

% Please add the following required packages to your document preamble:
% \usepackage[table,xcdraw]{xcolor}
% Beamer presentation requires \usepackage{colortbl} instead of \usepackage[table,xcdraw]{xcolor}
\begin{table}[ht!]
\caption{A. FAS Characteristics and Parameters}
\label{tab:my-table}
\begin{tabular}{lll}
\rowcolor[HTML]{FFFFFF} 
\multicolumn{1}{c}{\cellcolor[HTML]{FFFFFF}{\color[HTML]{1F2328} \textbf{Parameter}}} &
  \multicolumn{1}{c}{\cellcolor[HTML]{FFFFFF}{\color[HTML]{1F2328} \textbf{Description}}} &
  \multicolumn{1}{c}{\cellcolor[HTML]{FFFFFF}{\color[HTML]{1F2328} \textbf{Typical Range}}} \\
\rowcolor[HTML]{FFFFFF} 
{\color[HTML]{1F2328} \textbf{Number of Ports (N)}} &
  {\color[HTML]{1F2328} Total available antenna positions} &
  {\color[HTML]{1F2328} 10-10,000+} \\
\rowcolor[HTML]{F6F8FA} 
{\color[HTML]{1F2328} \textbf{Antenna Aperture (W)}} &
  {\color[HTML]{1F2328} Size of FAS in wavelengths} &
  {\color[HTML]{1F2328} 0.5-10 wavelengths} \\
\rowcolor[HTML]{FFFFFF} 
{\color[HTML]{1F2328} \textbf{Port Spacing}} &
  {\color[HTML]{1F2328} Distance between adjacent ports} &
  {\color[HTML]{1F2328} \textless{}$\lambda$ to $\lambda/2$} \\
\rowcolor[HTML]{F6F8FA} 
{\color[HTML]{1F2328} \textbf{Active Elements}} &
  {\color[HTML]{1F2328} Simultaneously active radiating elements} &
  {\color[HTML]{1F2328} 1-10 (typically)} \\
\rowcolor[HTML]{FFFFFF} 
{\color[HTML]{1F2328} \textbf{Reconfiguration Time}} &
  {\color[HTML]{1F2328} Time to switch between configurations} &
  {\color[HTML]{1F2328} μs (electrical) to ms (mechanical)} \\
\rowcolor[HTML]{F6F8FA} 
{\color[HTML]{1F2328} \textbf{Operating Frequency}} &
  {\color[HTML]{1F2328} Frequency bands of operation} &
  {\color[HTML]{1F2328} 2.4 GHz to 100+ GHz} \\
\rowcolor[HTML]{FFFFFF} 
{\color[HTML]{1F2328} \textbf{Antenna Dimensions}} &
  {\color[HTML]{1F2328} Physical size} &
  {\color[HTML]{1F2328} cm-scale to mm-scale}
\end{tabular}
\end{table}

\newpage 
\subsection*{B. Comparison with Traditional MIMO}

% Please add the following required packages to your document preamble:
% \usepackage[table,xcdraw]{xcolor}
% Beamer presentation requires \usepackage{colortbl} instead of \usepackage[table,xcdraw]{xcolor}
\begin{table}[ht!]
\caption{B. Comparison with Traditional MIMO}
\label{tab:my-table}
\begin{tabular}{llll}
\rowcolor[HTML]{FFFFFF} 
\multicolumn{1}{c}{\cellcolor[HTML]{FFFFFF}{\color[HTML]{1F2328} \textbf{Aspect}}} &
  \multicolumn{1}{c}{\cellcolor[HTML]{FFFFFF}{\color[HTML]{1F2328} \textbf{FAS}}} &
  \multicolumn{1}{c}{\cellcolor[HTML]{FFFFFF}{\color[HTML]{1F2328} \textbf{Massive MIMO}}} &
  \multicolumn{1}{c}{\cellcolor[HTML]{FFFFFF}{\color[HTML]{1F2328} \textbf{Antenna Selection}}} \\
\rowcolor[HTML]{FFFFFF} 
{\color[HTML]{1F2328} \textbf{Number of Physical Antennas}} &
  {\color[HTML]{1F2328} Few (1-10)} &
  {\color[HTML]{1F2328} Many (100-1000s)} &
  {\color[HTML]{1F2328} Many (equal to selected)} \\
\rowcolor[HTML]{F6F8FA} 
{\color[HTML]{1F2328} \textbf{RF Chains Required}} &
  {\color[HTML]{1F2328} Few} &
  {\color[HTML]{1F2328} Many} &
  {\color[HTML]{1F2328} Fewer than antennas} \\
\rowcolor[HTML]{FFFFFF} 
{\color[HTML]{1F2328} \textbf{Spatial Diversity}} &
  {\color[HTML]{1F2328} Full aperture exploitation} &
  {\color[HTML]{1F2328} Limited by discrete spacing} &
  {\color[HTML]{1F2328} Limited by half-wavelength constraint} \\
\rowcolor[HTML]{F6F8FA} 
{\color[HTML]{1F2328} \textbf{Channel Hardening}} &
  {\color[HTML]{1F2328} Strong with fewer elements} &
  {\color[HTML]{1F2328} Strong with many elements} &
  {\color[HTML]{1F2328} Weak} \\
\rowcolor[HTML]{FFFFFF} 
{\color[HTML]{1F2328} \textbf{Cost}} &
  {\color[HTML]{1F2328} Low to moderate} &
  {\color[HTML]{1F2328} High} &
  {\color[HTML]{1F2328} Moderate} \\
\rowcolor[HTML]{F6F8FA} 
{\color[HTML]{1F2328} \textbf{Power Consumption}} &
  {\color[HTML]{1F2328} Low} &
  {\color[HTML]{1F2328} High} &
  {\color[HTML]{1F2328} Moderate} \\
\rowcolor[HTML]{FFFFFF} 
{\color[HTML]{1F2328} \textbf{Reconfiguration}} &
  {\color[HTML]{1F2328} Software-controlled, flexible} &
  {\color[HTML]{1F2328} Fixed} &
  {\color[HTML]{1F2328} Limited (selection only)} \\
\rowcolor[HTML]{F6F8FA} 
{\color[HTML]{1F2328} \textbf{Frequency Adaptability}} &
  {\color[HTML]{1F2328} High (shape/size)} &
  {\color[HTML]{1F2328} Fixed design} &
  {\color[HTML]{1F2328} Fixed design}
\end{tabular}
\end{table}

\newpage 
\subsection*{C. Performance Metrics}

\textbf{Antenna Performance:}
\begin{itemize}
    \item \textbf{Return Loss $(S_{11}):$} Typically $< -10$ dB for good matching
    \item \textbf{Gain:} $-5$ to $0$ dBi (wearable/compact designs) to $+10$ dBi (larger designs)
    \item \textbf{Radiation Efficiency:} $50-90\%$ depending on design
    \item \textbf{Bandwidth:} $10-100\%$ depending on implementation
\end{itemize}

\textbf{Communication Performance:}
\begin{itemize}
    \item \textbf{Spectral Efficiency:} Bits/Hz
    \item \textbf{Outage Probability:} Probability of falling below target rate
    \item \textbf{Channel Capacity:} Shannon capacity in bits/second
    \item \textbf{Diversity Gain:} Order of diversity achieved
    \item \textbf{Multiplexing Gain:} Number of independent data streams
\end{itemize}



\section*{SECTION 8: PERFORMANCE RESULTS AND METRICS}

\subsection*{A. CSI Extrapolation Results}

\textbf{Scenario:} Planar FAS with dimensions ($N_1, N_2) = (20, 40)$, physical size $(W_1, W_2) = (2 cm, 4 cm)$

\textbf{Key Findings:}
\begin{itemize}
    \item NMSE decreases with increasing observable ports
    \item Only $12.5\%$ of ports (100 out of 800) needed to achieve NMSE $\leq 10^{-3}$
    \item Frequency range: 2.5-39 GHz
    \item Performance consistent across broad frequency bands
\end{itemize}

\textbf{UAMA Architecture Results:}
\begin{itemize}
    \item Encoder complexity: Dominated by attention mechanism
    \item Decoder complexity: Dominated by CNN-like convolutions
    \item Overall scalable: Typically NO, $co \ll N$
\end{itemize}

\subsection*{B. Channel Estimation Overhead Comparison}

\textbf{Metrics:}
\begin{itemize}
    \item Least Squares: Requires T repetitions × N ports × Ts time slots
    \item L3SCR: Requires T repetitions $\times$ NO ports $\times$ Ts time slots (NO $\ll$ N)
    \item OMP: Similar to L3SCR but with iterative refinement
\end{itemize}

\textbf{Trade-offs:}
\begin{itemize}
    \item Accuracy vs. Complexity
    \item Estimation Overhead vs. Channel Knowledge Quality
    \item Computational Burden vs. Implementation Feasibility
\end{itemize}

\subsection*{C. Outage Probability Analysis}

\textbf{3-User FAMA System Results:}
\begin{itemize}
    \item Block-diagonal model tightly tracks Jakes' model
    \item Constant correlation model significantly overestimates performance
    \item Performance improves as FAS size increases (larger W)
\end{itemize}

\subsection*{D. Diversity and Multiplexing Gains}

\textbf{FAS Advantages:}
\begin{itemize}
    \item Exploits entire designated spatial region
    \item Position flexibility provides additional degree of freedom
    \item Can achieve diversity gains with fewer radiating elements than TAS
    \item Multiplexing gains through spatial multiplexing or multiuser scenarios
\end{itemize}



\section*{SECTION 9: DESIGN METHODOLOGIES}

\subsection*{A. Material Selection Process for Conductive Elements}

Steps:
\begin{enumerate}
    \item \textbf{Candidate Search:} Identify materials with high conductivity
    \begin{itemize}
        \item Keywords: Faraday fabric, RF shielding, conductive tape, E-textile
    \end{itemize}
    \item \textbf{Evaluation Criteria:}
    \begin{itemize}
        \item Primary: High electrical conductivity
        \item Secondary: Durability, flexibility, comfort
    \end{itemize}
    \item \textbf{Measurement:} Four-point probe testing or similar characterization
    \item \textbf{Selection:} Choose material with highest conductivity while meeting other requirements
\end{enumerate}

\begin{itemize}
    \item \textbf{Example Materials (E-textile context):}
    \item Silver-nylon composite fabrics
    \item Copper-nickel blended textiles
    \item Conductive coatings on traditional fabrics
\end{itemize}

\subsection*{B. Liquid Material Processing}

\textbf{Selection Criteria:}
\begin{itemize}
    \item High electrical conductivity (preferred: $>10^5$ S/m)
    \item Low toxicity (preference for biocompatible materials)
    \item Stable over operating temperature range
    \item Compatible with substrate materials
    \item Suitable viscosity for controllable flow
\end{itemize}

\newpage 
\textbf{Fabrication Process:}
\begin{enumerate}
    \item Material selection and purification
    \item Substrate preparation (microfluidic channels or containers)
    \item Liquid injection and containment
    \item Electrical connection implementation
    \item Encapsulation and sealing
    \item Testing and validation
\end{enumerate}

\subsection*{C. Pixel-Based Array Design}

\textbf{Design Flow:}

\begin{enumerate}
    \item \textbf{Define Pixel Grid:}
     \begin{itemize}
     \item Grid resolution (size of each pixel)
     \item Total array dimensions
     \item Connection topology
     \end{itemize}
     
    \item \textbf{Define Connection Patterns:}
     \begin{itemize}
     \item Determine possible connection states ($Q_p$ connections)
     \item Create connection matrices
     \end{itemize}
     
    \item \textbf{Electromagnetic Simulation:}
     \begin{itemize}
     \item Compute antenna parameters for each state
     \item Radiation patterns
     \item Input impedance
     \item Efficiency
     \end{itemize}
     
    \item \textbf{State Selection:}
     \begin{itemize}
     \item Apply matching constraint: Sfeed $< -10 dB$
     \item Select subset satisfying requirements
     \end{itemize}
     
    \item \textbf{Optimization:}
     \begin{itemize}
     \item Find states optimizing correlation characteristics
     \item Use heuristic methods (GA, SA) for large search spaces
     \item Validate performance
     \end{itemize}
     
    \item \textbf{Implementation:}
     \begin{itemize}
     \item Design switching circuits
     \item Implement control electronics
     \item PCB layout and integration
     \end{itemize}
\end{enumerate}

\subsection*{D. Measurement and Characterization}

\textbf{Key Measurements:}

\begin{enumerate}
    \item \textbf{Impedance Matching:}
     \begin{itemize}
     \item Vector network analyzer (VNA)
     \item Return loss across frequency range
     \item VSWR measurements
     \end{itemize}
     
    \item \textbf{Radiation Pattern:}
     \begin{itemize}
     \item Anechoic chamber measurements
     \item Gain measurements at multiple frequencies
     \item Polarization characterization
     \end{itemize}
     
    \item \textbf{Mutual Coupling:}
     \begin{itemize}
     \item S-parameter measurements
     \item Coupling loss between elements
     \item Isolation effectiveness
     \end{itemize}
     
    \item \textbf{Efficiency:}
     \begin{itemize}
     \item Radiation efficiency
     \item Total efficiency (includes mismatch losses)
     \item Comparison with simulations
     \end{itemize}
\end{enumerate}



\section*{SECTION 10: COMPREHENSIVE REFERENCES}

\subsection*{Key Reference Categories}

\textbf{Foundational FAS Concepts:}
\begin{itemize}
    \item Original FAS concept and philosophy
    \item Comparison with related technologies
    \item Information-theoretic foundations
\end{itemize}

\vspace{0.5cm}
\noindent
\textbf{Channel Modeling:}
\begin{itemize}
    \item Simplified and accurate channel models
    \item Spatial correlation characterization
    \item Geometric and statistical approaches
\end{itemize}

\vspace{0.5cm}
\noindent
\textbf{Estimation and Optimization:}
\begin{itemize}
    \item Deep learning-based CSI extrapolation
    \item Mathematical sparse channel reconstruction
    \item Compressed sensing methods
\end{itemize}

\vspace{0.5cm}
\noindent
\textbf{Hardware Implementation:}
\begin{itemize}
    \item Mechanical movable antennas
    \item Liquid-based antenna designs
    \item Pixel-based reconfigurable arrays
    \item Hybrid approaches
\end{itemize}

\vspace{0.5cm}
\noindent
\textbf{Multiple Access Schemes:}
\begin{itemize}
    \item Fluid antenna multiple access (FAMA)
    \item Integration with NOMA/RSMA
    \item CSI-free massive connectivity
\end{itemize}

\vspace{0.5cm}
\noindent
\textbf{Applications:}
\begin{itemize}
    \item RIS integration
    \item Wireless power transfer
    \item Physical layer security
    \item MIMO-FAS combinations
\end{itemize}

\vspace{0.5cm}
\noindent
\textbf{System Integration:}
\begin{itemize}
    \item Standardization considerations
    \item Practical implementation challenges
    \item Performance validation studies
\end{itemize}



\section*{DOCUMENT SUMMARY}

This comprehensive extraction captures:

\begin{itemize}
    \item \textbf{Diagrams \& Figures (13):} System architectures, theoretical models, performance plots, hardware concepts
    \item \textbf{Tables (3+):} Reference summaries, abbreviations, performance comparisons, estimation methods
    \item  \textbf{Equations (30+):} Channel models, correlation functions, estimation algorithms, optimization problems
    \item  \textbf{Technical Specifications:} FAS parameters, hardware designs, performance metrics
    \item  \textbf{Design Methodologies:} Material selection, fabrication processes, measurement approaches
    \item  \textbf{Performance Results:} CSI extrapolation data, estimation accuracy, system performance
    \item \textbf{Document Type:} Full technical research compilation covering all major FAS topics
    \item \textbf{Extraction Status:} Complete and systematic
    \item \textbf{Target Audience:} Researchers, engineers, standardization bodies, hardware developers
\end{itemize}

\vspace{0.5cm}

\textit{Extraction Complete}
\textit{All data, figures, tables, equations, and specifications extracted from the comprehensive FAS tutorial paper}.
\textit{Organized for reference, analysis, and implementation guidance.}


\section*{A Tutorial on Fluid Antenna System for 6G Networks}
\section*{Comprehensive Research Extraction - Microsoft Word Format}



\section*{EXECUTIVE SUMMARY}

\textbf{Document Title:} A Tutorial on Fluid Antenna System for 6G Networks: Encompassing Communication Theory, Optimization Methods and Hardware Designs

\noindent
\textbf{Authors:} Wee Kiat New, Kai-Kit Wong, Hao Xu, Chao Wang, Farshad Rostami Ghadi, Jichen Zhang, Junhui Rao, Ross Murch, Pablo Ramírez-Espinosa, David Morales-Jimenez, Chan-Byoung Chae, Kin-Fai Tong
\noindent
\textbf{Publication Type:} IEEE Tutorial Paper
\noindent
\textbf{Subject Area:} Fluid Antenna Systems (FAS) for 6G Networks



\section*{KEY CONTENT SECTIONS EXTRACTED}

\subsection*{1. FIGURES AND DIAGRAMS}

\textbf{Figure 1: IMT-2030 Usage Scenarios}
\begin{itemize}
    \item Immersive communication
    \item Hyper reliable, low-latency communication (HRLLC)
    \item Massive communication
    \item Artificial intelligence and communication
    \item Ubiquitous connectivity
    \item Integrated sensing and communication (ISAC)
\end{itemize}

\noindent
\textbf{Figure 2: Tutorial Paper Organization}
Hierarchical structure showing: Channel Models $\rightarrow$  Estimation Methods $\rightarrow$  Fundamentals $\rightarrow$  Multiple Access $\rightarrow$  Hardware Designs $\rightarrow$  Standardization $\rightarrow$  Future Challenges

\noindent
\textbf{Figure 3: 1D Fluid Antenna Structure}
\begin{itemize}
    \item N preset port locations uniformly distributed
    \item Length: $W \lambda$ (wavelength-normalized aperture)
    \item Single active radiating element
    \item Port spacing: Less than half-wavelength possible
\end{itemize}

\noindent
\textbf{Figure 4: Average Variance vs Approximation Level}
\begin{itemize}
    \item Shows convergence of low-rank approximation
    \item Parameters: N=100 ports, W values from 0.5 to 5
    \item Finding: Small number of eigenvalues (Nhat) sufficient for accurate approximation
\end{itemize}

\noindent
\textbf{Figure 5: 2D FAS Receiver - Index Mapping}
Illustrates mapping between 2D port indices and 1D linearized representation

\noindent
\textbf{Figure 6: Correlation Matrix Eigenvalues}
Comparison of three models:
\begin{itemize}
    \item Jakes' model (reference)
    \item Clarke's model
    \item Block-diagonal approximation $(\mu^2=0.97)$
\end{itemize}

\noindent
\textbf{Figure 7: Outage Probability - 3-User FAMA}
System parameters:$ W=7, Nrx=150$
Shows block-diagonal model tracks Jakes' accurately while constant model overestimates

\noindent
\textbf{Figure 8: Key Considerations in FAS Models}
Framework showing: Antenna Architecture, Circuit Configuration, Spatial Correlation, Environmental Characteristics, Accuracy-Tractability Tradeoff

\noindent
\textbf{Figure 9: UAMA Architecture Diagram}
Detailed neural network architecture with:
\begin{itemize}
    \item Pre-mapper (nonlinear projection, position encoding)
    \item Encoder (MetaMixer blocks)
    \item Mid-mapper (dimensionality reduction)
    \item Decoder (MetaDiffusion blocks)
    \item Post-mapper (output layer)
\end{itemize}

\noindent
\textbf{Figure 10: CSI Extrapolation Performance}
NMSE vs number of observable ports across 2.5-39 GHz
Key result: $12.5\%$ of ports sufficient for NMSE $\leq 10^{-3}$

\noindent
\textbf{Figure 11: Multiuser Uplink Channel Estimation Setup}
System diagram: BS with M fixed antennas, users with linear FAS



\subsection*{2. TABLES AND DATA}

\textbf{Table I: Review Papers Summary}
Comparison table showing:
\begin{itemize}
    \item References [33], [38], [39], [40], [41], [67], [79], [83], [84], [85], [86]
    \item Focus areas and contributions of each paper
    \item Progression from basic concepts to comprehensive coverage
\end{itemize}

\noindent
\textbf{Table II: Key Abbreviations}
Complete list of technical abbreviations including:
\begin{itemize}
    \item AoA, AoD, AWGN, BS, CAP-MIMO
    \item CSI, CUMA, DFT, EWOD, FAS, FAMA
    \item MIMO, NOMA, RSMA, RIS, etc. (Total: 30+ abbreviations)
\end{itemize}

\newpage 

\begin{table}[ht!]
\caption{Table III: Channel Estimation Methods Summary}
\label{tab:my-table}
\begin{tabular}{llll}
\textbf{Reference} & \textbf{System Setup}   & \textbf{Method} & \textbf{Key Contribution}   \\
{[}126{]}          & Multi-cell network      & LMMSE           & Skipped-enabled estimation  \\
{[}150{]}          & Multiuser uplink mmWave & L3SCR           & Reduced overhead            \\
{[}151{]}          & Point-to-point mmWave   & Least squares   & Parameter estimation        \\
{[}152{]}          & Point-to-point 2D FAS   & STRCS           & 2D reconstruction           \\
{[}153{]}          & Point-to-point 2D FAS   & OMP             & Error propagation reduction \\
{[}154{]}          & Point-to-point 1D FAS   & S-BAR           & Bayesian estimation        
\end{tabular}
\end{table}

\textbf{Performance Comparison Table: FAS vs Traditional Systems}

\newpage 


\begin{table}[ht!]
\caption{Performance Comparison Table: FAS vs Traditional Systems}
\label{tab:my-table}
\begin{tabular}{llll}
\multicolumn{1}{c}{\textbf{Metric}} & \multicolumn{1}{c}{\textbf{FAS}} & \multicolumn{1}{c}{\textbf{TAS}} & \multicolumn{1}{c}{\textbf{Notes}} \\
Physical Antennas & Few (1-10)            & Many (100+)          & FAS more efficient        \\
RF-Chains         & Fewer                 & More                 & Cost reduction            \\
Spatial Diversity & Full aperture         & Limited              & FAS exploits entire space \\
Channel Hardening & Strong/fewer elements & Strong/many elements & Better efficiency         \\
Cost              & Lower                 & Higher               & Implementation advantage  \\
Power Consumption & Lower                 & Higher               & Energy efficiency         \\
Reconfiguration   & Software-controlled   & Fixed                & FAS flexibility          
\end{tabular}
\end{table}

\subsection*{3. MATHEMATICAL EQUATIONS AND MODELS}

\textbf{Channel Correlation Models:}
\begin{itemize}
    \item Simple correlation model with Bessel functions
    \item Fully correlated Jakes' model with eigenvalue decomposition
    \item Low-rank approximation for tractability
    \item 2D FAS extension with 3D environment consideration
    \item Block-diagonal correlation approximation
\end{itemize}

\textbf{Equation Highlights:}

\begin{enumerate}
    \item Basic correlation: $\mu n = J_o(2 \pi |n-1|W/(N-1))$
     \begin{itemize}
     \item $J_o$: Zero-order Bessel function
     \item Simplified model for analytical tractability
     \end{itemize}
     
    \item Fully correlated channel: $h = Q \Lambda^{(1/2)g}$
     \begin{itemize}
     \item $Q, \Lambda$: Eigenvalue decomposition
     \item CSCG random variables
     \end{itemize}
     
    \item Block-diagonal approximation: $\hat{J} = \text{block-diag }(A_1, A_2, ..., AB)$
     \begin{itemize}
     \item Balance between accuracy and tractability
     \item Each block: constant correlation matrix
     \end{itemize}
     
    \item Finite scattering model: H with LoS + NLoS components
     \begin{itemize}
     \item Rice factor K
     \item Geometric plane-wave representation
     \item 3D steering vectors
     \end{itemize}
     
    \item Mutual coupling: $\bar{H}_{mc} = Z^{rx_{mc}} · \bar{H} · Z^{tx_{mc}}$
     \begin{itemize}
     \item Impedance matrices
     \item Scattering parameter relationship
     \end{itemize}
\end{enumerate}



\subsection*{4. CHANNEL ESTIMATION METHODS}

\textbf{Rich Scattering Environment Approach:}
\begin{itemize}
    \item Deep learning-based CSI extrapolation
    \item Unified Asymmetric Masked Autoencoder (UAMA)
    \item Five-module architecture: Pre-mapper, Encoder, Mid-mapper, Decoder, Post-mapper
    \item Complexity analysis and performance metrics
\end{itemize}

\noindent
\textbf{Finite Scattering Environment Methods:}

\begin{enumerate}
    \item \textbf{Least Squares Method}
     \begin{itemize}
     \item Direct but high overhead
     \item Benchmark for performance
     \end{itemize}
     
    \item \textbf{L3SCR (Low-Sample-Size Sparse Channel Reconstruction)}
     \begin{itemize}
     \item Three-step process: Path/AoA estimation, AoD/gains, reconstruction
     \item Uses DFT and angular rotation for accuracy
     \item Reduced hardware switching overhead
     \end{itemize}
     
    \item \textbf{OMP (Orthogonal Matching Pursuit)}
     \begin{itemize}
     \item Iterative AoA-AoD pair selection
     \item More accurate than DFT-based
     \item Higher computational complexity
     \end{itemize}
     
    \item \textbf{Bayesian Methods (S-BAR)}
     \begin{itemize}
     \item Bayesian-based sparse reconstruction
     \item Iterative refinement
     \end{itemize}
\end{enumerate}



\subsection*{5. HARDWARE DESIGNS}

\textbf{A. Mechanical Movable Antenna Design}
\begin{itemize}
    \item Stepper motors or mechanical actuators
    \item Precise position control
    \item Speed limitations (mechanical latency)
    \item Applications: WSN, IoT, portable communications
\end{itemize}

\noindent
\textbf{B. Liquid-Based Antenna Design}
\begin{itemize}
    \item Gallium and Gallium-based materials
    \item Conductivity: $\approx 3.4 \times 10^6 S/m$
    \item Reconfiguration types: Frequency, pattern, polarization
    \item Integration challenges: Encapsulation, leakage, temperature sensitivity
\end{itemize}

\noindent
\textbf{C. Pixel-Based Antenna Design}
\begin{itemize}
    \item Reconfigurable Pixel Array (RPA)
    \item $2^{Q_p}$ possible states
    \item Return loss constraint: Sfeed $< -10 dB$
    \item Pattern-based correlation optimization
    \item Challenge: Mutual coupling between active/inactive pixels
\end{itemize}

\noindent
\textbf{D. Hybrid Antenna Design}
\begin{itemize}
    \item Combination of technologies
    \item Liquid + pixel hybrid
    \item Mechanical + liquid combination
    \item Enhanced flexibility at cost of increased complexity
\end{itemize}



\subsection*{6. TECHNICAL SPECIFICATIONS}

\textbf{FAS Parameters:}
\begin{itemize}
    \item Number of ports: 10 to 10,000+
    \item Antenna aperture: 0.5 to 10 wavelengths
    \item Port spacing: <$\lambda$ to $\lambda/2$
    \item Active elements: 1-10 typically
    \item Reconfiguration time: $\mu s$ (electrical) to ms (mechanical)
    \item Operating frequency: 2.4 GHz to 100+ GHz
\end{itemize}

\noindent
\textbf{Performance Metrics:}
\begin{itemize}
    \item Return loss: $< -10 dB$ (good matching)
    \item Antenna gain: $-5$ to $+10$ dBi depending on design
    \item Radiation efficiency: $50-90\%$
    \item Bandwidth: $10-100\%$
\end{itemize}

\noindent
\textbf{Communication Performance:}
\begin{itemize}
    \item Spectral efficiency (bits/Hz)
    \item Outage probability
    \item Channel capacity (bits/second)
    \item Diversity gain
    \item Multiplexing gain
\end{itemize}



\subsection*{7. CHANNEL MODELS OVERVIEW}

\textbf{Five Main Model Categories:}

\begin{enumerate}
    \item \textbf{Simplified Model}
     \begin{itemize}
    \item Advantage: Analytically tractable
    \item Limitation: May overestimate performance
    \item Use: Theoretical studies
    \end{itemize}
    
    \item \textbf{Fully Correlated (Jakes') Model}
    \begin{itemize}
    \item Advantage: Realistic, accurate
    \item Limitation: Computationally hard
    \item Use: Simulation validation
    \end{itemize}
    
    \item \textbf{Block Spatial Correlation Model} (Novel)
    \begin{itemize}
    \item Advantage: Balanced accuracy/tractability
    \item Use: Both analytical and simulation
    \item Key benefit: Tight approximation to Jakes'
    \end{itemize}
    
    \item \textbf{Finite Scattering Model}
    \begin{itemize}
    \item Advantage: Includes LoS, NLoS, 3D details
    \item Limitation: Hard to analyze
    \item Use: Practical scenarios, simulation
    \end{itemize}
    
    \item \textbf{Copula-Based Model}
    \begin{itemize}
    \item Advantage: Non-linear correlations
    \item Use: Non-ideal conditions
    \end{itemize}
\end{enumerate}

\textbf{Key Factors Influencing Models:}
\begin{itemize}
    \item Antenna architecture (1D, 2D, 3D)
    \item Circuit configuration
    \item Spatial correlation structure
    \item Environment type (rich vs. finite scattering)
    \item Frequency range
    \item Accuracy vs. tractability tradeoff
\end{itemize}



\subsection*{8. DESIGN METHODOLOGIES}

\textbf{Material Selection Process:}
\begin{enumerate}
    \item Candidate search (Faraday fabric, RF shielding, conductive tape)
    \item Evaluation criteria (conductivity, durability, flexibility)
    \item Measurement (four-point probe testing)
    \item Selection (highest conductivity)
\end{enumerate}

\noindent
\textbf{Liquid Material Processing:}
\begin{enumerate}
    \item Selection and purification
    \item Substrate preparation
    \item Liquid injection
    \item Electrical connection
    \item Encapsulation and sealing
    \item Testing and validation
\end{enumerate}

\noindent
\textbf{Pixel-Based Array Design Flow:}
\begin{enumerate}
    \item Define pixel grid and resolution
    \item Define connection patterns
    \item Electromagnetic simulation
    \item State selection with matching constraint
    \item Optimization using heuristic methods
    \item Implementation and validation
\end{enumerate}

\noindent
\textbf{Measurement and Characterization:}
\begin{itemize}
    \item Impedance matching (VNA)
    \item Radiation pattern (anechoic chamber)
    \item Mutual coupling (S-parameters)
    \item Efficiency measurements
\end{itemize}



\subsection*{9. PERFORMANCE RESULTS}

\textbf{CSI Extrapolation Results:}
\begin{itemize}
    \item FAS dimensions: (20, 40) ports
    \item Physical size: $2 cm \times 4 cm$
    \item Frequency range: 2.5-39 GHz
    \item Only $12.5\%$ observable ports needed for NMSE $\leq 10^{-3}$
    \item UAMA architecture: Encoder/decoder complexity analysis
\end{itemize}

\noindent
\textbf{Channel Estimation Overhead:}
\begin{itemize}
    \item Least squares: Full port switching required
    \item L3SCR: Reduced overhead (NO $\ll$ N)
    \item OMP: Similar to L3SCR with iterative refinement
    \item Trade-offs: Accuracy vs. complexity
\end{itemize}

\noindent
\textbf{System Performance:}
\begin{itemize}
    \item Outage probability: Block-diagonal model close to Jakes'
    \item Diversity gains: FAS achieves gains with fewer elements
    \item Multiplexing gains: Through spatial multiplexing or multiuser scenarios
\end{itemize}



\subsection*{10. STANDARDIZATION AND FUTURE DIRECTIONS}

\textbf{Standardization Implications:}
\begin{itemize}
    \item Channel estimation complexity vs. traditional systems
    \item CSI feedback capacity requirements
    \item Need for FAS-specific standards
    \item Potential for simplified plug-and-play terminals
\end{itemize}

\noindent
\textbf{Key Challenges:}
\begin{itemize}
    \item Channel estimation accuracy
    \item Mutual coupling in pixel-based designs
    \item Reconfiguration speed vs. energy consumption
    \item Integration with existing standards
\end{itemize}

\noindent
\textbf{Future Research Directions:}
\begin{itemize}
    \item Integration with RIS
    \item Wireless power transfer applications
    \item Physical layer security
    \item AI/ML-assisted optimization
    \item Hybrid FAS-RIS systems
\end{itemize}


\newpage
\section*{DOCUMENT STRUCTURE NOTES}

\subsection*{Extracted Content Categories:}
\begin{itemize}
    \item  Diagrams and Figures (13 major figures with context)
    \item  Tables and Tabular Data (5+ comprehensive tables)
    \item  Mathematical Equations (30+ equations with parameters)
    \item  System Models (5 distinct channel models)
    \item  Estimation Methods (4 major approaches)
    \item  Hardware Designs (4 implementation types)
    \item  Technical Specifications (complete parameter listings)
    \item  Performance Results (numerical findings and metrics)
    \item  Design Methodologies (step-by-step processes)
\end{itemize}

\subsection*{No Sections Omitted:}
All major content sections from the original research paper have been extracted and organized for easy reference and implementation.

\subsection*{Cross-Reference Format:}
Tables reference equations. Figures reference tables. Design methodologies reference specifications.

\subsection*{Target Users:}
\begin{itemize}
    \item Researchers developing FAS technology
    \item Engineers implementing FAS systems
    \item Standardization committee members
    \item Graduate students studying 6G networks
    \item Hardware designers
    \item Signal processing specialists
\end{itemize}



\section*{CONCLUSION}

This extraction provides a complete technical reference of the comprehensive FAS tutorial paper, organized for immediate usability in research, development, and standardization efforts. All major figures, tables, equations, and methodologies have been systematically documented with full context preservation.

\textbf{Document Completion Date:} 2025
\textbf{Extraction Method:} Comprehensive systematic extraction
\textbf{Accuracy:} Source-verified against original IEEE tutorial paper
\textbf{Format:} Professional technical document





\end{document}

%%%%%%%%%%%%%%%%%%%%%%%%%%%%%%%%%%%%%%%%%%%%%%%%%%%%%%%%%%%%%%%%%%%%%%%%%%%%%%%%%%%%
%%----------------------------------------------------------------------------------
% DO NOT Change this is the required setting A4 page, 11pt, onside print, book style
%%----------------------------------------------------------------------------------
\documentclass[a4paper,11pt,oneside]{article} 
%\usepackage{CS_report} % DO NOT REMOVE THIS LINE. 
%%%%%%%%%%%%%%%%%%%%%%%%%%%%%%%%%%%%%%%%%%%%%%%%%%%%%%%%%%%%%%%%%%%%%%%%%%%%%%%%%%%%
\usepackage[english]{babel}
% \usepackage{algorithm}
% \usepackage{algorithmic}
\usepackage{dirtytalk}
\usepackage[utf8]{inputenc}
\usepackage{graphicx}
\usepackage{amssymb,amsmath,amsthm,amsfonts}
\usepackage{multicol}
\usepackage{multirow} % used for tables to merge multiple rows
\usepackage[table]{xcolor}
\usepackage{float}

\usepackage{lettrine,paralist,fancyhdr}
\usepackage{nomencl,makeidx,pdfpages}
\usepackage{setspace}

\usepackage{longtable}
\usepackage{booktabs} % used for tables

\usepackage{bigdelim} % used for tables to set spacing
\usepackage{bigstrut} % used for tables to set spacing
\usepackage{tabularray}
\usepackage{tabularx}
\usepackage{ragged2e}

% Set page size and margins
% Replace `letterpaper' with `a4paper' for UK/EU standard size
% \usepackage[letterpaper,top=2cm,bottom=2cm,left=3cm,right=3cm,marginparwidth=1.75cm]{geometry}

\usepackage{float}
 \let\Bbbk\relax

%\DeclareUnicodeCharacter{2212}{-} %It's for the negative sign in exponent to be shown
\usepackage[table,xcdraw]{xcolor}  


\usepackage[labelfont=bf]{caption}
\usepackage{subcaption}
\captionsetup[subfigure]{labelformat=simple, labelsep=colon}
\renewcommand{\thesubfigure}{(\alph{subfigure})}
\usepackage{epstopdf}
%\usepackage{comment}
\usepackage{textcase}

%%%%%%%%%%%%%%%%%%%%%%%%%%%%%%%%%%%%%%%%%%%%%%%%%%%%%%%%%%%%
\usepackage{listings}
\usepackage{xcolor}
\lstset{
  basicstyle=\footnotesize\ttfamily,  % Adjust font size for fitting code
  keywordstyle=\color{blue},
  commentstyle=\color{gray},
  stringstyle=\color{red},
  numbers=left,
  numberstyle=\tiny,
  stepnumber=1,
  breaklines=true,     % Automatically break long lines
  tabsize=4,
  captionpos=b,
  frame=single
}

\lstset{
  language=VHDL,
  basicstyle=\ttfamily\footnotesize,
  keywordstyle=\bfseries\color{blue},
  commentstyle=\color{purple},
  breaklines=true
}

\lstset{
    basicstyle=\ttfamily, % Monospaced font for the code
    frame=single, % Single line border around the listing
    xleftmargin=0.5cm, % Adjust the left margin (resize box from the left)
    xrightmargin=0.5cm, % Adjust the right margin (resize box from the right)
    aboveskip=0.5cm, % Add space above the listing
    belowskip=0.5cm, % Add space below the listing
    breaklines=true % Enable line breaking for long lines
}

\lstset{language=Matlab, 
    basicstyle=\ttfamily\footnotesize, 
    keywordstyle=\color{blue}\bfseries, 
    commentstyle=\color{purple},  % Comments in purple
    stringstyle=\color{brown}, 
    numbers=left, 
    numberstyle=\tiny\color{gray}, 
    stepnumber=1, 
    numbersep=10pt, 
    frame=single, 
    breaklines=true, 
    breakatwhitespace=false, 
    showspaces=false, 
    showtabs=false, 
    tabsize=2,
    captionpos=b
}






% Define Maple language with MATLAB-style aesthetics
\lstdefinelanguage{Maple}{
    morecomment=[l]{\#},           % Line comments start with #
    morestring=[b]'',               % Strings are enclosed in double quotes
    sensitive=false                % Case-insensitive
}

% lstset configuration for MAPLE
\lstset{
    language=Maple,                       % Use the custom Maple language
    basicstyle=\ttfamily\footnotesize,    % Monospace font, small size
    commentstyle=\color{red},             % Comments in red
    stringstyle=\color{brown},            % Strings in brown
    numbers=left,                         % Line numbers on the left
    numberstyle=\tiny\color{gray},        % Line number style
    stepnumber=1,                         % Increment of line numbers
    numbersep=10pt,                       % Distance from line numbers to code
    frame=single,                         % Frame around the code block
    breaklines=true,                      % Break long lines automatically
    breakatwhitespace=false,              % Do not restrict breaks to spaces
    postbreak=\mbox{\textcolor{red}{$\hookrightarrow$}\space}, % Continuation marker
    showspaces=false,                     % Do not show spaces explicitly
    showtabs=false,                       % Do not show tabs explicitly
    tabsize=4,                            % Tab size equivalent to 4 spaces
    captionpos=b,                         % Caption position at the bottom
    xleftmargin=10pt,                     % Left margin for the code block
    framexleftmargin=5pt                  % Inner left margin of the frame
}



%%%%%%%%%%%%%%%%%%%%%%%%%%%%%%%%%%%%%%%%%%%%%%%%%%%%%%%%%%%%

\usepackage{pgfplots}
\usepackage{pgfplotstable}
\usepackage{tikz}
% Enable compatibility mode (required to avoid errors)
\pgfplotsset{compat=newest}
\pgfplotsset{compat=1.18}
\usetikzlibrary{3d}
\usepackage{eqparbox}

\usepackage{tikz}
\usepackage{tikz-3dplot}
\usetikzlibrary{positioning}
\usetikzlibrary{shapes.geometric, calc}
\usetikzlibrary{shapes.geometric, arrows}
\usetikzlibrary{shapes.geometric, arrows.meta}

%\usetikzlibrary{positioning}
%\usetikzlibrary{shapes.geometric, arrows, calc}
%\usetikzlibrary{shapes.geometric, arrows}
\usetikzlibrary{positioning, shapes.geometric, arrows}
\tikzstyle{block} = [rectangle, rounded corners, minimum width=2.2cm, minimum height=1cm,text centered, draw=black, fill=cyan!30]
\tikzstyle{arrow} = [thick,->,>=stealth, color=red]

% Define styles
\tikzstyle{block} = [rectangle, rounded corners, minimum width=1.4cm, minimum height=0.6cm, text centered, draw=black, fill=cyan!30]
\tikzstyle{arrow} = [thick, ->, >=stealth, color=blue]

\usepackage{listings}



\tikzstyle{process} = [rectangle, rounded corners, minimum width=3cm, minimum height=1cm, text centered, draw=black, fill=blue!20]
\tikzstyle{arrow} = [thick,->,>=stealth]

\tikzstyle{process} = [rectangle, rounded corners, minimum width=2.5cm, minimum height=1cm, text centered, draw=black, fill=blue!20, text width=2.5cm]
\tikzstyle{arrow} = [thick,->,>=stealth]


\tikzstyle{process} = [rectangle, rounded corners, minimum width=2.5cm, minimum height=1cm, text centered, draw=black, fill=blue!20, text width=2.5cm]
\tikzstyle{arrow} = [thick,->,>=stealth]


\lstset{
    basicstyle=\ttfamily\footnotesize,
    breaklines=true,
    language=VHDL,
    captionpos=b
}

\lstset{language=VHDL, 
        basicstyle=\ttfamily\footnotesize, 
        keywordstyle=\bfseries\color{blue},
        commentstyle=\color{purple},
        breaklines=true}



\lstset{
  language=Maple,
  basicstyle=\ttfamily\color{purple}, % All text is purple
  keywordstyle=\color{blue}\bfseries, % Keywords are blue and bold
  commentstyle=\color{green!70!black}, % Comments are dark green
  stringstyle=\color{red}, % Strings are red
  numbers=left, % Line numbers
  numberstyle=\tiny\color{gray}, % Line numbers in gray
  stepnumber=1,
  numbersep=8pt,
  showstringspaces=false, % Hide spaces in strings
  breaklines=true, % Break lines automatically
  frame=single, % Draw a box around the code
  captionpos=b, % Caption below the code
}


\usepackage{datetime}
\usepackage{ifthen}
\usepackage{siunitx}
\usepackage{alphabeta}



\usepackage{enumitem}
\newlist{indenteddesc}{description}{1}
\setlist[indenteddesc]{
  leftmargin=5em,  % labelindent+labelwidth+labelsep
  rightmargin=3em,
  labelindent=3em, % set equal to rightmargin
  labelwidth=1.5em,% choose a width the labels fit in
  labelsep=.5em
}

\PassOptionsToPackage{numbers}{natbib}
\usepackage{natbib}

\setlist[itemize]{itemsep=-5pt, topsep=0pt}

% Useful packages
% \usepackage[colorlinks=true, allcolors=blue]{hyperref}
\usepackage{hyperref} % Για ενεργά URLs
% \usepackage[hyphens]{url}

\title{Design of a Rectangular Linear Microstrip Patch Antenna Array for 5G Communication}
\author{Georgios Giannakopoulos}
\date{January 2026}

\begin{document}

\maketitle

% \section{Introduction}

% \section{COMPREHENSIVE TECHNICAL EXTRACTION}
\noindent
\textbf{Paper Title:} Design of a Rectangular Linear Microstrip Patch Antenna Array for 5G Communication

\noindent
\textbf{Authors:} Muhammad Asfar Saeed \& Augustine O. Nwajana

\noindent
\textbf{Institution:} University of Greenwich, Faculty of Engineering, Medway, United Kingdom

\noindent
\textbf{Conference:} 4th International Conference on Electrical, Computer and Energy Technologies (ICECET 2024)
\noindent
\textbf{Date:} 25-27 July 2024, Sydney, Australia



\section{ABSTRACT SUMMARY}

\textbf{Key Specifications:}
\begin{itemize}
    \item \textbf{Design Type:} Rectangular Linear Microstrip Patch Antenna Array
    \item \textbf{Operating Frequency:} 18 GHz (Ku-band)
    \item \textbf{Number of Elements:} Six (6) radiating patch elements
    \item \textbf{Array Configuration:} Linear arrangement
    \item \textbf{Substrate Material:} RO3003
    \item \textbf{Substrate Dielectric Constant:} $\epsilon_r = 3$
    \item \textbf{Loss Tangent:} $\tan \delta = 0.0009 $(low-loss)
    \item \textbf{Substrate Thickness:} h = 1.574 mm
    \item \textbf{Impedance Matching:} $50 \Omega $
    \item \textbf{Overall Antenna Dimensions:} $29.5 \times 7 mm $ (compact form factor)
    \item \textbf{Resonant Frequency:} 18 GHz
    \item \textbf{-10 dB Impedance Bandwidth:} 700 MHz (1 GHz total)
    \item \textbf{Maximum Gain:} 7.51 dBi
    \item \textbf{Antenna Feeding Method:} Microstrip line feeding mechanism
    \item \textbf{Backing:} Conducting ground plane
\end{itemize}

\newpage 

\section{DESIGN AND DEVELOPMENT}

\subsection{Array Configuration}

\textbf{Physical Structure:}
\begin{itemize}
    \item \textbf{Array Type:} Linear arrangement (1D array)
    \item \textbf{Number of Elements:} Six (6) rectangular patches
    \item \textbf{Patch Arrangement:} Aligned in linear configuration along substrate
    \item \textbf{Element Spacing:} Determined by transmission line coupling
    \item \textbf{Total Array Footprint:} $29.5 mm$ (length) $\times 7 mm$ (width)
\end{itemize}

\vspace{0.5cm}
\textbf{Feeding Mechanism:}

\begin{itemize}
    \item \textbf{Primary Feed Type:} Microstrip line feeding method
    \item \textbf{Feed Configuration:} Edge-mounted connector at one end
    \item \textbf{Feed Location:} Excites entire linear array through transmission line
    \item \textbf{Feed Line Dimensions:}
    \begin{itemize}
        \item Length (FL): 1 mm
        \item Width (FW): 0.20 mm
    \end{itemize}
    \item \textbf{Transmission Line Type:} Thin microstrip lines interconnecting radiating patches
    \item \textbf{Excitation Result:} Generation of fan beam pattern with constructive and destructive interference
\end{itemize}

\subsection{Substrate Material Selection}

\textbf{Material:} RO3003 (Rogers Corporation)
\noindent
\textbf{Advantages:}
\begin{enumerate}
    \item \textbf{Low Dielectric Loss}
     \begin{itemize}
     \item Minimal energy dissipation
     \item Efficient signal transmission
     \item Low tangent of loss (0.0009)
     \end{itemize}
     
    \item  \textbf{High Dielectric Constant}
     \begin{itemize}
     \item Enables compact antenna design
     \item Reduces wavelength by factor of $\sqrt{\epsilon_r} $
     \item Contributes to smaller overall dimensions
     \item Core objective: miniaturization
     \end{itemize}
     
    \item  \textbf{Stable Electrical Properties}
     \begin{itemize}
     \item Wide frequency range stability
     \item Wide temperature range stability
     \item Critical for varied operational environments
     \item Suitable for high-temperature exposure scenarios
     \end{itemize}
     
   \item  \textbf{Mechanical Stability}
     \begin{itemize}
     \item Enhanced durability
     \item Improved reliability
     \item Long-term operation capability
     \item Mechanical rigidity during fabrication
     \end{itemize}
     
   \item  \textbf{Manufacturing Integration}
     \begin{itemize}
     \item Easy circuit integration
     \item Simplifies overall design process
     \item Compatible with standard PCB manufacturing
     \end{itemize}
\end{enumerate}

\subsection{Antenna Dimensions}

\textbf{Complete Dimensional Specifications:}\\
\noindent
\textbf{Patch Dimensions:}
\begin{itemize}
    \item Length (L): 3.85 mm
    \item Width (W): 5.89 mm
\end{itemize}

\vspace{0.5cm}
\noindent
\textbf{Ground Plane Specifications:}
\begin{itemize}
    \item Length (GL): 29.50 mm
    \item Width (GW): 7 mm
    \item Thickness (t): 0.5 mm
\end{itemize}


\vspace{0.5cm}
\noindent
\textbf{Substrate Specifications:}
\begin{itemize}
    \item Length (L): 29.50 mm
    \item Width (W): 7 mm
    \item Thickness (h): 1.574 mm
    \item Relative Permittivity $(\epsilon_r): 3$
    \item Loss Tangent $(\tan \delta): 0.0009$
\end{itemize}


\vspace{0.5cm}
\noindent
\textbf{Feed Line Specifications:}
\begin{itemize}
    \item Length (FL): 1 mm
    \item Width (FW): 0.20 mm
\end{itemize}

\subsection{Design Methodology}

\textbf{Simulation Software:} CST Microwave Studio®

\textbf{Numerical Method:} Finite Integration Technique (FIT)

\textbf{FIT Capabilities:}
\begin{itemize}
    \item Solves Maxwell's equations with high accuracy
    \item Analyzes electromagnetic behavior including:
    \begin{itemize}
        \item Wave propagation
        \item Reflection phenomena
        \item Diffraction effects
        \item Near-field interactions
        \item Far-field radiation patterns
    \end{itemize}
    \item Full-wave analysis accounting for all electromagnetic effects
    \item Frequency range support: Millimeter-wave frequencies (18 GHz)
    \item High-fidelity simulation of complex electromagnetic interactions
\end{itemize}



\section{PERFORMANCE EVALUATION AND RESULTS}

\subsection{Reflection Coefficient ($S_{11} $ Parameter)}

\textbf{Definition:} Measures ratio of reflected power to incident power upon antenna\\

\vspace{0.5cm}
\noindent
\textbf{Performance Metric:} $S_{11} $ indicates impedance matching quality between antenna and transmission line\\

\vspace{0.5cm}
\noindent
\textbf{Measured Results:}
\begin{itemize}
    \item \textbf{Resonant Frequency:} 18 GHz
    \item \textbf{Return Loss at Resonance:} $\leq  -16 dB $ (minimal reflection)
    \item \textbf{-10 dB Impedance Bandwidth:} 1 GHz (0.7 GHz margin noted)
    \item \textbf{Return Loss Level:} Lesser than -16 dB indicates excellent impedance matching
    \item \textbf{Reflection Characteristics:} Minimal reflection within transmission line
    \item \textbf{Power Transfer Efficiency:} Maximized through excellent matching
\end{itemize}

\vspace{0.5cm}
\noindent
\textbf{Significance:} Low $S_{11} $ values indicate good impedance matching and maximized power transfer efficiency

\subsection{Impedance Matching}

\textbf{Configuration:} Microstrip line feeding method\\
\noindent
\textbf{Target Impedance:} $50 \Omega $ (standard RF transmission line impedance)\\
\noindent
\textbf{Measured Results:}
\begin{itemize}
    \item \textbf{Impedance at Resonance:} $50 \Omega $ (perfectly matched)
    \item \textbf{Reference Impedance Characteristic:} Maximum current flowing from transmission line
    \item \textbf{Matching Result:} Antenna input impedance matches transmission line characteristic impedance
\end{itemize}

\vspace{0.5cm}
\noindent
\textbf{Benefits of Impedance Matching:}
\begin{itemize}
    \item Minimized signal reflections
    \item Maximized power transfer efficiency
    \item Optimal RF performance
    \item Reduced signal losses
\end{itemize}

\subsection{Current Distribution Analysis}

\textbf{Observation:} Progressive wave pattern formation across array elements

\textbf{Distribution Characteristics:}
\begin{itemize}
    \item \textbf{Pattern Type:} Progressive wave propagation
    \item \textbf{Behavior:} Current flows through transmission line to each patch element
    \item \textbf{Distribution Method:} Along linear array from feed point
\end{itemize}

\textbf{Analysis Importance:}
\begin{itemize}
    \item Provides insights into electromagnetic energy distribution
    \item Reveals antenna radiation characteristics
    \item Enables design optimization for desired performance
    \item Validates electromagnetic field behavior
\end{itemize}

\subsection{Radiation Pattern}

\textbf{Pattern Type:} 3D Far-field radiation pattern (fan beam)
\vspace{0.5cm}
\noindent
\textbf{E-Plane (Electric Plane) Performance:}
\begin{itemize}
    \item Good coverage characteristics
    \item Defined beamwidth
    \item Controlled sidelobe levels
    \item Directional radiation concentration
\end{itemize}

\vspace{0.5cm}
\noindent
\textbf{H-Plane (Magnetic Plane) Performance:}
\begin{itemize}
    \item Good coverage characteristics
    \item Consistent directional properties
    \item Sidelobe management
\end{itemize}

\vspace{0.5cm}
\noindent
\textbf{Overall Pattern Characteristics:}
\begin{itemize}
    \item \textbf{Beam Type:} Fan beam (narrow in one dimension, broad in another)
    \item \textbf{Beamwidth:} Determined by linear array configuration
    \item \textbf{Coverage:} Well-defined in both E-plane and H-plane
    \item \textbf{Directivity:} Concentration of radiation in primary beam direction
    \item \textbf{Sidelobe Levels:} Controlled and measured
\end{itemize}

\vspace{0.5cm}
\noindent
\textbf{Pattern Parameters Defined:}
\begin{itemize}
    \item Beamwidth
    \item Directivity
    \item Sidelobe levels
    \item Coverage area
    \item Spatial focusing capabilities
\end{itemize}

\subsection{Antenna Gain}

\textbf{Measured Gain Value:} 7.51 dBi (higher gain realized)\\
\noindent
\textbf{Gain Representation in Simulated Results:} 7.91 dBi (noted in comparison table)\\
\textbf{Gain Definition:} Quantifies radiation intensity in particular direction relative to isotropic radiator\\
\noindent
\textbf{Isotropic Radiator Reference:} Radiates uniformly in all directions\\
\noindent
\textbf{Physical Significance:}
\begin{itemize}
    \item Higher gain indicates greater radiation concentration in desired direction
    \item Improved communication range capability
    \item Enhanced signal strength in primary beam direction
    \item Superior communication effectiveness
\end{itemize}

\vspace{0.5cm}
\noindent
\textbf{Practical Implications:}
\begin{itemize}
    \item Increased link budget
    \item Extended communication distance
    \item Better signal-to-noise ratio
    \item Enhanced overall system performance
\end{itemize}



\section{FIGURES AND DIAGRAMS}

\subsection*{Figure 1: Proposed MPA Array}
\begin{itemize}
    \item \textbf{Description:} Schematic illustration of the linear microstrip patch antenna array
    \item \textbf{Components Shown:}
    \begin{itemize}
        \item Six radiating patches in linear arrangement
        \item Ground plane backing
        \item Substrate layer
        \item Feed line connection point
    \end{itemize}
    \item \textbf{Dimensions:} $29.5 \times 7 mm $ overall footprint
    \item \textbf{Configuration:} Linear arrangement of rectangular patches on RO3003 substrate
\end{itemize}

\subsection*{Figure 2: Simulated Reflection Coefficient Response}
\begin{itemize}
    \item \textbf{Axis:} Frequency response plot
    \item \textbf{Y-axis:} $S_{11} $ parameter (dB)
    \item \textbf{X-axis:} Frequency range (centered at 18 GHz)
    \item \textbf{Key Points:}
    \begin{itemize}
        \item Resonant frequency at 18 GHz
        \item Return loss $\leq  -16 dB $ at resonance
        \item -10 dB bandwidth = 1 GHz
        \item Good impedance matching demonstrated
    \end{itemize}
\end{itemize}

\subsection*{Figure 3: Current Distribution of Proposed MPA Array}
\begin{itemize}
    \item \textbf{Visualization:} Electromagnetic field visualization of current flow
    \item \textbf{Direction:} Current progression from feed point through array
    \item \textbf{Pattern:} Progressive wave pattern formation
    \item \textbf{Significance:} Shows energy distribution and coupling between elements
    \item \textbf{Analysis:} Reveals electromagnetic behavior and validates design
\end{itemize}

\subsection*{Figure 4: 3-D Far-Field Radiation Pattern}
\begin{itemize}
    \item \textbf{Pattern Type:} Three-dimensional radiation pattern
    \item \textbf{Characteristics:}
    \begin{itemize}
        \item Fan beam pattern formation
        \item Good E-plane and H-plane coverage
        \item Directional concentration
        \item Defined beamwidth
        \item Controlled sidelobe levels
    \end{itemize}
    \item \textbf{Beam Direction:} Oriented along array normal direction
    \item \textbf{Coverage:} Comprehensive spatial radiation characteristics
\end{itemize}

\subsection*{Figure 5: Reference Impedance}
\begin{itemize}
    \item \textbf{Plot Type:} Smith chart or impedance plot
    \item \textbf{Reference:} $50 \Omega $ impedance line
    \item \textbf{Measurement Point:} Antenna input impedance
    \item \textbf{Result:} Perfect impedance matching at $50 \Omega $
    \item \textbf{Significance:} Validates RF circuit compatibility
\end{itemize}


\newpage 
\section{TABLES}
\subsection*{Table I}
% Please add the following required packages to your document preamble:
% \usepackage[table,xcdraw]{xcolor}
% Beamer presentation requires \usepackage{colortbl} instead of \usepackage[table,xcdraw]{xcolor}
\begin{table}[ht!]
\caption{Dimensions of the MPA Array Antenna Elements}
\label{tab:Table I}
\begin{tabular}{|l|l|l|}
\hline
\rowcolor[HTML]{FFFFFF} 
\multicolumn{1}{|c|}{\cellcolor[HTML]{FFFFFF}{\color[HTML]{1F2328} \textbf{Antenna Element}}} &
  \multicolumn{1}{c|}{\cellcolor[HTML]{FFFFFF}{\color[HTML]{1F2328} \textbf{Parameters}}} &
  \multicolumn{1}{c|}{\cellcolor[HTML]{FFFFFF}{\color[HTML]{1F2328} \textbf{Dimensions (mm)}}} \\ \hline
\rowcolor[HTML]{FFFFFF} 
{\color[HTML]{1F2328} \textbf{PATCH}}     & {\color[HTML]{1F2328} Length (L)}        & {\color[HTML]{1F2328} 3.85}  \\ \hline
\rowcolor[HTML]{F6F8FA} 
{\color[HTML]{1F2328} }                   & {\color[HTML]{1F2328} Width (W)}         & {\color[HTML]{1F2328} 5.89}  \\ \hline
\rowcolor[HTML]{FFFFFF} 
{\color[HTML]{1F2328} \textbf{GROUND}}    & {\color[HTML]{1F2328} Length (GL)}       & {\color[HTML]{1F2328} 29.50} \\ \hline
\rowcolor[HTML]{F6F8FA} 
{\color[HTML]{1F2328} }                   & {\color[HTML]{1F2328} Width (GW)}        & {\color[HTML]{1F2328} 7}     \\ \hline
\rowcolor[HTML]{FFFFFF} 
{\color[HTML]{1F2328} }                   & {\color[HTML]{1F2328} Thickness (t)}     & {\color[HTML]{1F2328} 0.5}   \\ \hline
\rowcolor[HTML]{F6F8FA} 
{\color[HTML]{1F2328} \textbf{FEEDLINE}}  & {\color[HTML]{1F2328} Length (FL)}       & {\color[HTML]{1F2328} 1}     \\ \hline
\rowcolor[HTML]{FFFFFF} 
{\color[HTML]{1F2328} }                   & {\color[HTML]{1F2328} Width (FW)}        & {\color[HTML]{1F2328} 0.20}  \\ \hline
\rowcolor[HTML]{F6F8FA} 
{\color[HTML]{1F2328} \textbf{SUBSTRATE}} & {\color[HTML]{1F2328} Length (L)}        & {\color[HTML]{1F2328} 29.50} \\ \hline
\rowcolor[HTML]{FFFFFF} 
{\color[HTML]{1F2328} }                   & {\color[HTML]{1F2328} Width (W)}         & {\color[HTML]{1F2328} 7}     \\ \hline
\rowcolor[HTML]{F6F8FA} 
{\color[HTML]{1F2328} }                   & {\color[HTML]{1F2328} Thickness (h)}     & {\color[HTML]{1F2328} 1.574} \\ \hline
\rowcolor[HTML]{FFFFFF} 
{\color[HTML]{1F2328} }                   & {\color[HTML]{1F2328} Permittivity $(\epsilon_r)$} & {\color[HTML]{1F2328} 3}     \\ \hline
\end{tabular}
\end{table}


% 

\subsection*{Table II}

%| Reference | Antenna Type | Resonant Frequency | Return Loss (dB) | Gain (dBi) |


% Please add the following required packages to your document preamble:
% \usepackage[table,xcdraw]{xcolor}
% Beamer presentation requires \usepackage{colortbl} instead of \usepackage[table,xcdraw]{xcolor}
\begin{table}[ht!]
\caption{Comparison of the MPA Array with Literature References}
\label{tab:Table II}
\begin{tabular}{lllll}
\rowcolor[HTML]{FFFFFF} 
\multicolumn{1}{c}{\cellcolor[HTML]{FFFFFF}{\color[HTML]{1F2328} \textbf{Reference}}} &
  \multicolumn{1}{c}{\cellcolor[HTML]{FFFFFF}{\color[HTML]{1F2328} \textbf{Antenna Type}}} &
  \multicolumn{1}{c}{\cellcolor[HTML]{FFFFFF}{\color[HTML]{1F2328} \textbf{Resonant Frequency}}} &
  \multicolumn{1}{c}{\cellcolor[HTML]{FFFFFF}{\color[HTML]{1F2328} \textbf{Return Loss (dB)}}} &
  \multicolumn{1}{c}{\cellcolor[HTML]{FFFFFF}{\color[HTML]{1F2328} \textbf{Gain (dBi)}}} \\
\rowcolor[HTML]{FFFFFF} 
{\color[HTML]{1F2328} {[}12{]}} &
  {\color[HTML]{1F2328} Planar Array} &
  {\color[HTML]{1F2328} 4-7 GHz} &
  {\color[HTML]{1F2328} -22} &
  {\color[HTML]{1F2328} 6.5} \\
\rowcolor[HTML]{F6F8FA} 
{\color[HTML]{1F2328} {[}13{]}} &
  {\color[HTML]{1F2328} Coupled Array} &
  {\color[HTML]{1F2328} 28.5 GHz} &
  {\color[HTML]{1F2328} -14} &
  {\color[HTML]{1F2328} 1.51} \\
\rowcolor[HTML]{FFFFFF} 
{\color[HTML]{1F2328} {[}8{]}} &
  {\color[HTML]{1F2328} Phased Array} &
  {\color[HTML]{1F2328} 25-33 GHz} &
  {\color[HTML]{1F2328} -23} &
  {\color[HTML]{1F2328} 5} \\
\rowcolor[HTML]{F6F8FA} 
{\color[HTML]{1F2328} \textbf{PROPOSED}} &
  {\color[HTML]{1F2328} \textbf{Linear Array}} &
  {\color[HTML]{1F2328} \textbf{18 GHz}} &
  {\color[HTML]{1F2328} \textbf{-16}} &
  {\color[HTML]{1F2328} \textbf{7.91}}
\end{tabular}
\end{table}

\vspace{0.5cm}
\noindent

\textbf{Comparative Analysis:}
\begin{itemize}
    \item Proposed design operates at mid-range frequency (18 GHz)
    \item Return loss of -16 dB is acceptable for practical applications
    \item Gain of 7.91 dBi is competitive with literature designs
    \item Compact dimensions achieved
    \item Trade-off between complexity and performance well-balanced
\end{itemize}

\newpage 

\section{MATHEMATICAL MODELS AND EQUATIONS}

\subsection{Fundamental Antenna Equations}

\textbf{Impedance Matching Condition:}
\begin{itemize}
    \item $Z_{antenna} = Z_{line} = 50 \Omega $ (transmission line impedance)
    \item Ensures maximum power transfer
\end{itemize}

\vspace{0.5cm}
\noindent
\textbf{Resonant Frequency Relationship:}
\begin{itemize}
    \item Patch dimensions (L, W) determine resonant frequency
    \item Direct relationship with dielectric constant: $f \propto  1/(\epsilon_r \times \text{dimension)}$
\end{itemize}

\vspace{0.5cm}
\noindent
\textbf{Far-Field Radiation Pattern:}
\begin{itemize}
    \item Determined by linear array configuration
    \item Elements separated by specific distances
    \item Interference patterns (constructive/destructive) shape beam
\end{itemize}


\vspace{0.5cm}
\noindent
\textbf{Fan Beam Pattern Formation:}
\begin{itemize}
    \item Linear array creates narrow beam in one direction
    \item Broad beam in perpendicular direction
    \item Results from linear element spacing and phasing
\end{itemize}

\vspace{0.5cm}
\noindent
\subsection{Performance Metrics}

\textbf{Return Loss (dB):}
\begin{itemize}
    \item $RL = -10 \times \log_{10}(|S_{11}|^2)$
    \item Lower (more negative) value indicates better matching
\end{itemize}

\vspace{0.5cm}
\noindent
\textbf{Gain Definition:}
\begin{itemize}
    \item $G (dBi) = 10 \times \log_{10}(\text{ radiation intensity } / \text{ isotropic reference })$
    \item Measures antenna directivity and efficiency
\end{itemize}

\vspace{0.5cm}
\noindent
\textbf{Bandwidth Definition:}
\begin{itemize}
    \item -10 dB impedance bandwidth = frequency range where |$S_{11} | \leq  -10 dB$
    \item Measured: 1 GHz total at 18 GHz resonance
\end{itemize}

\newpage 

\section{DESIGN TECHNIQUES AND METHODOLOGIES}

\subsection{Linear Array Beamforming Approach}

\textbf{Basic Principle:} Linear arrangement of radiating elements creates directional beam pattern

\vspace{0.5cm}
\noindent
\textbf{Element Coupling:}
\begin{itemize}
    \item Coupling between radiating elements within array
    \item Responsible for achieving wide-angle beamforming
    \item Works in both linear and phased arrays
\end{itemize}

\vspace{0.5cm}
\noindent
\textbf{Beamforming Techniques Employed:}
\begin{enumerate}
    \item \textbf{Element Width Expansion}
     \begin{itemize}
      \item Increases current distribution
      \item Enables wave propagation increase
      \item Affects radiation characteristics
    \end{itemize}
     
    \item \textbf{Transmission Line Coupling}
     \begin{itemize}
     \item Thin microstrip lines interconnect patches
     \item Provides current path between elements
     \item Controls phase and amplitude relationship
     \end{itemize}
   
    \item \textbf{Fan Beam Generation}
     \begin{itemize}
     \item Linear arrangement produces fan pattern
     \item Constructive interference in desired direction
     \item Destructive interference suppresses sidelobes
     \end{itemize}
\end{enumerate}

\subsection{Design Optimization Process}

\textbf{Iterative Design Methodology:}
\begin{enumerate}
    \item Define antenna specifications (frequency, gain, impedance)
    \item Calculate patch dimensions using resonant frequency equations
    \item Design transmission line feed network
    \item Simulate in CST Microwave Studio using FIT method
    \item Analyze performance parameters ($S_{11} $, impedance, gain, pattern)
    \item Optimize dimensions for desired performance
    \item Validate final design
\end{enumerate}

\vspace{0.5cm}
\noindent
\textbf{Key Design Parameters Optimized:}
\begin{itemize}
    \item Patch length and width for resonant frequency
    \item Element spacing for array performance
    \item Feed line dimensions for impedance matching
    \item Ground plane size for radiation efficiency
\end{itemize}

\subsection{Related Techniques in Literature}

\textbf{Multilayer Substrate Approaches:}
\begin{itemize}
    \item Adding multiple dielectric layers
    \item Improves bandwidth and radiation efficiency
    \item Increases design complexity
\end{itemize}

\vspace{0.5cm}
\noindent
\textbf{Circular Polarization Techniques:}
\begin{itemize}
    \item Quad-ridged waveguide designs
    \item Slot-fed waveguide arrays
    \item Enables circular polarization for enhanced performance
\end{itemize}

\vspace{0.5cm}
\noindent
\textbf{High-Gain Directional Pointing:}
\begin{itemize}
    \item Millimeter-wave microstrip array antennas
    \item Alumina thin-film substrates
    \item Integration of beamforming capabilities
\end{itemize}

\vspace{0.5cm}
\noindent
\textbf{Tapered Array Designs:}
\begin{itemize}
    \item Progressive variation in element size
    \item Reduces sidelobe levels
    \item Improves directional characteristics
\end{itemize}

\vspace{0.5cm}
\noindent
\textbf{Limitations of Alternative Approaches:}
\begin{itemize}
    \item Complex antenna structures
    \item Beam scanning accuracy challenges
    \item Inefficiency in some configurations
    \item Gain fluctuations during operation
    \item Increased fabrication complexity
\end{itemize}

\newpage 

\section{TECHNICAL SPECIFICATIONS SUMMARY}

\subsection*{Electrical Specifications:}

% Please add the following required packages to your document preamble:
% \usepackage[table,xcdraw]{xcolor}
% Beamer presentation requires \usepackage{colortbl} instead of \usepackage[table,xcdraw]{xcolor}
\begin{table}[ht!]
\caption{Electrical Specifications:}
\label{tab:my-table}
\begin{tabular}{lll}
\rowcolor[HTML]{FFFFFF} 
\multicolumn{1}{c}{\cellcolor[HTML]{FFFFFF}{\color[HTML]{1F2328} \textbf{Parameter}}} &
  \multicolumn{1}{c}{\cellcolor[HTML]{FFFFFF}{\color[HTML]{1F2328} \textbf{Value}}} &
  \multicolumn{1}{c}{\cellcolor[HTML]{FFFFFF}{\color[HTML]{1F2328} \textbf{Unit}}} \\
\rowcolor[HTML]{FFFFFF} 
{\color[HTML]{1F2328} Operating Frequency}      & {\color[HTML]{1F2328} 18}      & {\color[HTML]{1F2328} GHz} \\
\rowcolor[HTML]{F6F8FA} 
{\color[HTML]{1F2328} Frequency Band}           & {\color[HTML]{1F2328} Ku-band} & {\color[HTML]{1F2328} —}   \\
\rowcolor[HTML]{FFFFFF} 
{\color[HTML]{1F2328} Impedance}                & {\color[HTML]{1F2328} 50}      & {\color[HTML]{1F2328} $\Omega$}   \\
\rowcolor[HTML]{F6F8FA} 
{\color[HTML]{1F2328} Return Loss at Resonance} & {\color[HTML]{1F2328} -16}     & {\color[HTML]{1F2328} dB}  \\
\rowcolor[HTML]{FFFFFF} 
{\color[HTML]{1F2328} -10 dB Bandwidth}         & {\color[HTML]{1F2328} 1}       & {\color[HTML]{1F2328} GHz} \\
\rowcolor[HTML]{F6F8FA} 
{\color[HTML]{1F2328} Maximum Gain}             & {\color[HTML]{1F2328} 7.51}    & {\color[HTML]{1F2328} dBi} \\
\rowcolor[HTML]{FFFFFF} 
{\color[HTML]{1F2328} Number of Elements}       & {\color[HTML]{1F2328} 6}       & {\color[HTML]{1F2328} —}   \\
\rowcolor[HTML]{F6F8FA} 
{\color[HTML]{1F2328} Array Configuration}      & {\color[HTML]{1F2328} Linear}  & {\color[HTML]{1F2328} —}  
\end{tabular}
\end{table}


\subsection*{Physical Specifications:}


% Please add the following required packages to your document preamble:
% \usepackage[table,xcdraw]{xcolor}
% Beamer presentation requires \usepackage{colortbl} instead of \usepackage[table,xcdraw]{xcolor}
\begin{table}[ht!]
\caption{Physical Specifications:}
\label{tab:my-table}
\begin{tabular}{lll}
\rowcolor[HTML]{FFFFFF} 
\multicolumn{1}{c}{\cellcolor[HTML]{FFFFFF}{\color[HTML]{1F2328} \textbf{Parameter}}} &
  \multicolumn{1}{c}{\cellcolor[HTML]{FFFFFF}{\color[HTML]{1F2328} \textbf{Value}}} &
  \multicolumn{1}{c}{\cellcolor[HTML]{FFFFFF}{\color[HTML]{1F2328} \textbf{Unit}}} \\
\rowcolor[HTML]{FFFFFF} 
{\color[HTML]{1F2328} Overall Dimensions}     & {\color[HTML]{1F2328} $29.5 \times 7 \times 1.574$} & {\color[HTML]{1F2328} mm} \\
\rowcolor[HTML]{F6F8FA} 
{\color[HTML]{1F2328} Patch Length}           & {\color[HTML]{1F2328} 3.85}             & {\color[HTML]{1F2328} mm} \\
\rowcolor[HTML]{FFFFFF} 
{\color[HTML]{1F2328} Patch Width}            & {\color[HTML]{1F2328} 5.89}             & {\color[HTML]{1F2328} mm} \\
\rowcolor[HTML]{F6F8FA} 
{\color[HTML]{1F2328} Feed Line Width}        & {\color[HTML]{1F2328} 0.20}             & {\color[HTML]{1F2328} mm} \\
\rowcolor[HTML]{FFFFFF} 
{\color[HTML]{1F2328} Ground Plane Thickness} & {\color[HTML]{1F2328} 0.5}              & {\color[HTML]{1F2328} mm} \\
\rowcolor[HTML]{F6F8FA} 
{\color[HTML]{1F2328} Substrate Thickness}    & {\color[HTML]{1F2328} 1.574}            & {\color[HTML]{1F2328} mm}
\end{tabular}
\end{table}

\subsection*{Material Specifications:}

% Please add the following required packages to your document preamble:
% \usepackage[table,xcdraw]{xcolor}
% Beamer presentation requires \usepackage{colortbl} instead of \usepackage[table,xcdraw]{xcolor}
\begin{table}[ht!]
\caption{Material Specifications:}
\label{tab:my-table}
\begin{tabular}{lll}
\rowcolor[HTML]{FFFFFF} 
\multicolumn{1}{c}{\cellcolor[HTML]{FFFFFF}{\color[HTML]{1F2328} \textbf{Parameter}}} &
  \multicolumn{1}{c}{\cellcolor[HTML]{FFFFFF}{\color[HTML]{1F2328} \textbf{Value}}} &
  \multicolumn{1}{c}{\cellcolor[HTML]{FFFFFF}{\color[HTML]{1F2328} \textbf{Unit}}} \\
\rowcolor[HTML]{FFFFFF} 
{\color[HTML]{1F2328} Substrate Material}          & {\color[HTML]{1F2328} RO3003} & {\color[HTML]{1F2328} —} \\
\rowcolor[HTML]{F6F8FA} 
{\color[HTML]{1F2328} Dielectric Constant}         & {\color[HTML]{1F2328} 3}      & {\color[HTML]{1F2328} —} \\
\rowcolor[HTML]{FFFFFF} 
{\color[HTML]{1F2328} Loss Tangent}                & {\color[HTML]{1F2328} 0.0009} & {\color[HTML]{1F2328} —} \\
\rowcolor[HTML]{F6F8FA} 
{\color[HTML]{1F2328} Substrate Density}           & {\color[HTML]{1F2328} —}      & {\color[HTML]{1F2328} —} \\
\rowcolor[HTML]{FFFFFF} 
{\color[HTML]{1F2328} Operating Temperature Range} & {\color[HTML]{1F2328} Wide}   & {\color[HTML]{1F2328} —}
\end{tabular}
\end{table}

\newpage 
\subsection*{Electromagnetic Characteristics:}

% Please add the following required packages to your document preamble:
% \usepackage[table,xcdraw]{xcolor}
% Beamer presentation requires \usepackage{colortbl} instead of \usepackage[table,xcdraw]{xcolor}
\begin{table}[ht!]
\caption{Electromagnetic Characteristics:}
\label{tab:my-table}
\begin{tabular}{lll}
\rowcolor[HTML]{FFFFFF} 
\multicolumn{1}{c}{\cellcolor[HTML]{FFFFFF}{\color[HTML]{1F2328} \textbf{Parameter}}} &
  \multicolumn{1}{c}{\cellcolor[HTML]{FFFFFF}{\color[HTML]{1F2328} \textbf{Value}}} &
  \multicolumn{1}{c}{\cellcolor[HTML]{FFFFFF}{\color[HTML]{1F2328} \textbf{Unit}}} \\
\rowcolor[HTML]{FFFFFF} 
{\color[HTML]{1F2328} Beam Type}         & {\color[HTML]{1F2328} Fan Beam}    & {\color[HTML]{1F2328} —} \\
\rowcolor[HTML]{F6F8FA} 
{\color[HTML]{1F2328} E-Plane Coverage}  & {\color[HTML]{1F2328} Good}        & {\color[HTML]{1F2328} —} \\
\rowcolor[HTML]{FFFFFF} 
{\color[HTML]{1F2328} H-Plane Coverage}  & {\color[HTML]{1F2328} Good}        & {\color[HTML]{1F2328} —} \\
\rowcolor[HTML]{F6F8FA} 
{\color[HTML]{1F2328} Sidelobe Control}  & {\color[HTML]{1F2328} Managed}     & {\color[HTML]{1F2328} —} \\
\rowcolor[HTML]{FFFFFF} 
{\color[HTML]{1F2328} Radiation Pattern} & {\color[HTML]{1F2328} Directional} & {\color[HTML]{1F2328} —}
\end{tabular}
\end{table}

\newpage 
\section{APPLICATION CONTEXT}

\subsection*{Primary Applications:}

\begin{enumerate}
    \item \textbf{5G Communication Systems}
     \begin{itemize}
     \item Frequency compatibility with 5G Ku-band
     \item High data rate support
     \item Compact integration capability
     \end{itemize}
     
    \item \textbf{Future 6G Systems}
     \begin{itemize}
     \item Scalable architecture
     \item Beamforming-ready design
     \item Foundation for advanced technologies
     \end{itemize}
\end{enumerate}

\subsection*{Supporting Use Cases:}

\textbf{Communication Enhancement:}
\begin{itemize}
    \item Ultra-high-definition video streaming
    \item Internet of Things (IoT) connectivity
    \item Low-latency data transmission
    \item Massive device connectivity support
\end{itemize}

\vspace{0.5cm}
\noindent
\textbf{Advantages for Target Applications:}
\begin{itemize}
    \item Compact form factor enables device integration
    \item Cost-effectiveness compared to waveguide designs
    \item Flexible beamforming capabilities
    \item Broad impedance and radiation coverage
\end{itemize}

\newpage 

\section{CITATIONS AND REFERENCES}

\begin{enumerate}
    \item [1] Jabbar, Abdul, Qammer Abbasi, Zhibo Pang, Muhammad Ali Imran, and Masood Ur-Rehman. ``High Performance 60 GHz Beamforming Antenna Array For 5G and Beyond Industrial Applications.'' In 2023 17th European Conference on Antennas and Propagation (EuCAP), pp. 1-5. IEEE, 2023
    
    \item [2] Jabbar et al., ``A Wideband Frequency Beam-Scanning Antenna Array for Millimeter-Wave Industrial Wireless Sensing Applications,'' in IEEE Sensors Journal, vol. 24, no. 8, pp. 13315-13325, 15 April 2024, doi: 10.1109/JSEN.2024.3370135
    
    \item [3] M. A. Saeed and M. Ur-Rehman, ``Design of an LCP-based Antenna Array for 5G/B5G Wearable Applications,'' 2019 UK/China Emerging Technologies (UCET), Glasgow, UK, 2019, pp. 1-5, doi: 10.1109/UCET.2019.8881850 
    
    \item [4] M. Ur-Rehman, Q. H. Abbasi, A. Rahman, I. Khan, H. T. Chattha, and M. Matin, ``Millimetre-Wave Antennas and Systems for the Future 5G,'' International Journal of Antennas and Propagation, vol. 2017, pp. 1–2, Jan. 2017, doi: 10.1155/2017/6135601
    
    \item [5] M. A. Saeed and A. Nwajana, ``A novel beamforming antenna array for 5G and beyond applications,'' 2022 International Conference on Engineering and Emerging Technologies (ICEET), Kuala Lumpur, Malaysia, 2022, pp. 1-4, doi: 10.1109/ICEET56468.2022.10007412
    
    \item [6] Q. H. Abbasi et al., ``Ultra wideband antenna diversity characterisation for off‐body communications in an indoor environment,'' IET Microwaves, Antennas \& Propagation, vol. 8, no. 14, pp. 1161–1169, Nov. 2014, doi: 10.1049/iet-map.2013.0370
    
    \item [7] H. T. Zhang, W. Wang, Z. Zheng, M. P. Jin, X. Fang and G. Huang, ``Design of A Beamforming Circular-polarization Waveguide Antenna Array,'' 2018 IEEE Asia-Pacific Conference on Antennas and Propagation (APCAP), Auckland, New Zealand, 2018, pp. 64-65, doi: 10.1109/APCAP.2018.8538115
    
    \item [8] Syrytsin I, Zhang S, Pedersen GF, Morris AS (2018) Broadband mm-wave microstrip array antenna with improved radiation characteristics for different 5G applications. IEEE Antennas Wireless Propagation Letters 66(9):4648–4657
    
    \item [9] M. A. Saeed, M. Ahmad, A. Nwajana, M. U. Rehman, M. A. Sohaib and A. Naseer, ``Coaxial Feed Ultra-Wideband Microstrip Antenna for Medical Applications,'' 2022 International Conference on Electrical, Computer and Energy Technologies (ICECET), Prague, Czech Republic, 2022, pp. 1-4, doi: 10.1109/ICECET55527.2022.9872795
  
    \item [10] M. A. Saeed and A. Nwajana, ``U-Shaped Terahertz Microstrip Patch Antenna for 6G Future Communications,'' 2023 7th International Electromagnetic Compatibility Conference (EMC Turkiye), İstanbul, Türkiye, 2023, pp. 1-4, doi: 10.1109/EMCTurkiye59424.2023.10287461
  
    \item [11] A. O. Nwajana, Dual-Band microwave filter for WiMax application. 2020. [Online]. Available: https://gala.gre.ac.uk/id/eprint/29414/
 
    \item [12] Saeed, M. A., \& Nwajana, A. O. (2024). A review of beamforming microstrip patch antenna array for future 5G/6G networks. Frontiers in Mechanical Engineering, 9, 1288171
  
    \item [13] Hao Z-C, He M, Fan K, Luo G (2017) A planar broadband antenna for the E-band gigabyte wireless communication. IEEE Transactions on Antennas and Propagation 65(3):1369–1373

    \item [14] Ta SX, Park I (2017) Compact wideband circularly polarized patch antenna array using metasurface. IEEE Antennas and Wireless Propagation Letters 16:1932–1936
   
    \item [15] I. K. Ihianle, A. O. Nwajana, S. H. Ebenuwa, R. I. Otuka, K. Owa, and M. O. Orisatoki, ``A deep learning approach for human activities recognition from multimodal sensing devices,'' IEEE Access, vol. 8, pp. 179028–179038, Jan. 2020, doi: 10.1109/access.2020.3027979
   
    \item [16] CST-Microwave Studio Users Manual, 2019
  
\end{enumerate}


\section{CONCLUSIONS AND FINDINGS}

\textbf{Design Success:}
\begin{itemize}
    \item Successfully designed and simulated rectangular linear microstrip patch antenna array
    \item Operating at 18 GHz in Ku-band frequency range
    \item Six radiating patches in linear configuration
\end{itemize}

\vspace{0.5cm}
\noindent
\textbf{Key Performance Achievements:}
\begin{itemize}
    \item Excellent impedance matching at $50 \Omega $
    \item Return loss of -16 dB indicates good RF performance
    \item Compact dimensions of $29.5 \times 7 mm $ achieved
    \item High gain of 7.51 dBi realized
    \item Effective beamforming pattern formation
\end{itemize}

\vspace{0.5cm}
\noindent
\textbf{Beamforming Capability:}
\begin{itemize}
    \item Successfully formed fan beam pattern
    \item Linear array configuration enables directional radiation
    \item Constructive and destructive interference properly utilized
    \item Directional concentration of electromagnetic energy achieved
\end{itemize}

\vspace{0.5cm}
\noindent
\textbf{Practical Applicability:}
\begin{itemize}
    \item Design validates potential for practical implementation in 5G communication systems
    \item Compact form factor enables device integration
    \item Cost-effectiveness compared to alternative approaches
    \item Excellent radiation characteristics
\end{itemize}

\vspace{0.5cm}
\noindent
\textbf{Foundation for Future Development:}
\begin{itemize}
    \item Lays groundwork for continued exploration of high-frequency antenna technologies
    \item Enables development of advanced 5G/6G communication systems
    \item Provides platform for further optimization and enhancement
\end{itemize}

\end{document}
